%  LaTeX support: latex@mdpi.com 
%  For support, please attach all files needed for compiling as well as the log file, and specify your operating system, LaTeX version, and LaTeX editor.

%=================================================================
\documentclass[entropy,article,submit,moreauthors,pdftex]{Definitions/mdpi} 

% For posting an early version of this manuscript as a preprint, you may use "preprints" as the journal and change "submit" to "accept". The document class line would be, e.g., \documentclass[preprints,article,accept,moreauthors,pdftex]{mdpi}. This is especially recommended for submission to arXiv, where line numbers should be removed before posting. For preprints.org, the editorial staff will make this change immediately prior to posting.

%--------------------
% Class Options:
%--------------------
%----------
% journal
%----------
% Choose between the following MDPI journals:
% acoustics, actuators, addictions, admsci, adolescents, aerospace, agriculture, agriengineering, agronomy, ai, algorithms, allergies, analytica, animals, antibiotics, antibodies, antioxidants, appliedchem, applmech, applmicrobiol, applnano, applsci, arts, asi, atmosphere, atoms, audiolres, automation, axioms, batteries, bdcc, behavsci, beverages, biochem, bioengineering, biologics, biology, biomechanics, biomedicines, biomedinformatics, biomimetics, biomolecules, biophysica, biosensors, biotech, birds, bloods, brainsci, buildings, businesses, cancers, carbon, cardiogenetics, catalysts, cells, ceramics, challenges, chemengineering, chemistry, chemosensors, chemproc, children, civileng, cleantechnol, climate, clinpract, clockssleep, cmd, coatings, colloids, compounds, computation, computers, condensedmatter, conservation, constrmater, cosmetics, crops, cryptography, crystals, curroncol, cyber, dairy, data, dentistry, dermato, dermatopathology, designs, diabetology, diagnostics, digital, disabilities, diseases, diversity, dna, drones, dynamics, earth, ebj, ecologies, econometrics, economies, education, ejihpe, electricity, electrochem, electronicmat, electronics, encyclopedia, endocrines, energies, eng, engproc, entropy, environments, environsciproc, epidemiologia, epigenomes, fermentation, fibers, fire, fishes, fluids, foods, forecasting, forensicsci, forests, fractalfract, fuels, futureinternet, futuretransp, futurepharmacol, futurephys, galaxies, games, gases, gastroent, gastrointestdisord, gels, genealogy, genes, geographies, geohazards, geomatics, geosciences, geotechnics, geriatrics, hazardousmatters, healthcare, hearts, hemato, heritage, highthroughput, histories, horticulturae, humanities, hydrogen, hydrology, hygiene, idr, ijerph, ijfs, ijgi, ijms, ijns, ijtm, ijtpp, immuno, informatics, information, infrastructures, inorganics, insects, instruments, inventions, iot, j, jcdd, jcm, jcp, jcs, jdb, jfb, jfmk, jimaging, jintelligence, jlpea, jmmp, jmp, jmse, jne, jnt, jof, joitmc, jor, journalmedia, jox, jpm, jrfm, jsan, jtaer, jzbg, kidney, land, languages, laws, life, liquids, literature, livers, logistics, lubricants, machines, macromol, magnetism, magnetochemistry, make, marinedrugs, materials, materproc, mathematics, mca, measurements, medicina, medicines, medsci, membranes, metabolites, metals, metrology, micro, microarrays, microbiolres, micromachines, microorganisms, minerals, mining, modelling, molbank, molecules, mps, mti, nanoenergyadv, nanomanufacturing, nanomaterials, ncrna, network, neuroglia, neurolint, neurosci, nitrogen, notspecified, nri, nursrep, nutrients, obesities, oceans, ohbm, onco, oncopathology, optics, oral, organics, osteology, oxygen, parasites, parasitologia, particles, pathogens, pathophysiology, pediatrrep, pharmaceuticals, pharmaceutics, pharmacy, philosophies, photochem, photonics, physchem, physics, physiolsci, plants, plasma, pollutants, polymers, polysaccharides, proceedings, processes, prosthesis, proteomes, psych, psychiatryint, publications, quantumrep, quaternary, qubs, radiation, reactions, recycling, regeneration, religions, remotesensing, reports, reprodmed, resources, risks, robotics, safety, sci, scipharm, sensors, separations, sexes, signals, sinusitis, smartcities, sna, societies, socsci, soilsystems, solids, sports, standards, stats, stresses, surfaces, surgeries, suschem, sustainability, symmetry, systems, taxonomy, technologies, telecom, textiles, thermo, tourismhosp, toxics, toxins, transplantology, traumas, tropicalmed, universe, urbansci, uro, vaccines, vehicles, vetsci, vibration, viruses, vision, water, wevj, women, world 

%---------
% article
%---------
% The default type of manuscript is "article", but can be replaced by: 
% abstract, addendum, article, book, bookreview, briefreport, casereport, comment, commentary, communication, conferenceproceedings, correction, conferencereport, entry, expressionofconcern, extendedabstract, datadescriptor, editorial, essay, erratum, hypothesis, interestingimage, obituary, opinion, projectreport, reply, retraction, review, perspective, protocol, shortnote, studyprotocol, systematicreview, supfile, technicalnote, viewpoint, guidelines, registeredreport, tutorial
% supfile = supplementary materials

%----------
% submit
%----------
% The class option "submit" will be changed to "accept" by the Editorial Office when the paper is accepted. This will only make changes to the frontpage (e.g., the logo of the journal will get visible), the headings, and the copyright information. Also, line numbering will be removed. Journal info and pagination for accepted papers will also be assigned by the Editorial Office.

%------------------
% moreauthors
%------------------
% If there is only one author the class option oneauthor should be used. Otherwise use the class option moreauthors.

%---------
% pdftex
%---------
% The option pdftex is for use with pdfLaTeX. If eps figures are used, remove the option pdftex and use LaTeX and dvi2pdf.

%%%% ДЛЯ РУССКОГО ТЕКСТА закомментировать потом!
\usepackage{inputenc}
\usepackage[T2A,T1]{fontenc}
\usepackage[english,russian]{babel}
\usepackage{cmap}
%%%%


%=================================================================
% MDPI internal commands
\firstpage{1} 
\makeatletter 
\setcounter{page}{\@firstpage} 
\makeatother
\pubvolume{1}
\issuenum{1}
\articlenumber{0}
\pubyear{2021}
\copyrightyear{2020}
%\externaleditor{Academic Editor: Firstname Lastname} % For journal Automation, please change Academic Editor to "Communicated by"
\datereceived{} 
\dateaccepted{} 
\datepublished{} 
\hreflink{https://doi.org/} % If needed use \linebreak
%------------------------------------------------------------------
% The following line should be uncommented if the LaTeX file is uploaded to arXiv.org
%\pdfoutput=1

%=================================================================
% Add packages and commands here. The following packages are loaded in our class file: fontenc, inputenc, calc, indentfirst, fancyhdr, graphicx, epstopdf, lastpage, ifthen, lineno, float, amsmath, setspace, enumitem, mathpazo, booktabs, titlesec, etoolbox, tabto, xcolor, soul, multirow, microtype, tikz, totcount, changepage, paracol, attrib, upgreek, cleveref, amsthm, hyphenat, natbib, hyperref, footmisc, url, geometry, newfloat, caption

%=================================================================
%% Please use the following mathematics environments: Theorem, Lemma, Corollary, Proposition, Characterization, Property, Problem, Example, ExamplesandDefinitions, Hypothesis, Remark, Definition, Notation, Assumption
%% For proofs, please use the proof environment (the amsthm package is loaded by the MDPI class).

%=================================================================
% Full title of the paper (Capitalized)
\Title{Title}

% MDPI internal command: Title for citation in the left column
\TitleCitation{Title}

% Author Orchid ID: enter ID or remove command
\newcommand{\orcidauthorA}{0000-0001-5273-2471} % Add \orcidA{} behind the author's name
\newcommand{\orcidauthorB}{0000-0002-8736-0652} % Add \orcidB{} behind the author's name
\newcommand{\orcidauthorC}{0000-0002-4013-2329} % Add \orcidB{} behind the author's name

% Authors, for the paper (add full first names)
\Author{Konstantin Barkalov $^{1}$*\orcidA{}, Ilya Lebedev $^{1}$\orcidB{} and Victor Gergel $^{1}$\orcidC{}}

% MDPI internal command: Authors, for metadata in PDF
\AuthorNames{Konstantin Barkalov, Ilya Lebedev and Victor Gergel}

% MDPI internal command: Authors, for citation in the left column
\AuthorCitation{Barkalov, K.; Lebedev, I.; Gergel, V.}
% If this is a Chicago style journal: Lastname, Firstname, Firstname Lastname, and Firstname Lastname.

% Affiliations / Addresses (Add [1] after \address if there is only one affiliation.)
\address[1]{%
$^{1}$ \quad Department of Mathematical Software and Supercomputing Technologies, Lobachevsky University, 603950 Nizhni Novgorod, Russia; ilya.lebedev@itmm.unn.ru (I.L.); gergel@unn.ru (V.G.)}
%$^{2}$ \quad Affiliation 2; e-mail@e-mail.com}

% Contact information of the corresponding author
\corres{Correspondence: konstantin.barkalov@itmm.unn.ru}
%; Tel.: (optional; include country code; if there are multiple corresponding authors, add author initials) +xx-xxxx-xxx-xxxx (F.L.)}

% Current address and/or shared authorship
%\firstnote{Current address: Affiliation 3} 
%\secondnote{These authors contributed equally to this work.}
% The commands \thirdnote{} till \eighthnote{} are available for further notes

%\simplesumm{} % Simple summary

%\conference{} % An extended version of a conference paper

% Abstract (Do not insert blank lines, i.e. \\) 
\abstract{A single paragraph of about 200 words maximum. For research articles, abstracts should give a pertinent overview of the work. We strongly encourage authors to use the following style of structured abstracts, but without headings: (1) Background: place the question addressed in a broad context and highlight the purpose of the study; (2) Methods: describe briefly the main methods or treatments applied; (3) Results: summarize the article's main findings; (4) Conclusion: indicate the main conclusions or interpretations. The abstract should be an objective representation of the article, it must not contain results which are not presented and substantiated in the main text and should not exaggerate the main conclusions.}

% Keywords
\keyword{keyword 1; keyword 2; keyword 3 (List three to ten pertinent keywords specific to the article; yet reasonably common within the subject discipline.)} 

% The fields PACS, MSC, and JEL may be left empty or commented out if not applicable
%\PACS{J0101}
%\MSC{}
%\JEL{}

%%%%%%%%%%%%%%%%%%%%%%%%%%%%%%%%%%%%%%%%%%
% Only for the journal Diversity
%\LSID{\url{http://}}

%%%%%%%%%%%%%%%%%%%%%%%%%%%%%%%%%%%%%%%%%%
% Only for the journal Applied Sciences:
%\featuredapplication{Authors are encouraged to provide a concise description of the specific application or a potential application of the work. This section is not mandatory.}
%%%%%%%%%%%%%%%%%%%%%%%%%%%%%%%%%%%%%%%%%%

%%%%%%%%%%%%%%%%%%%%%%%%%%%%%%%%%%%%%%%%%%
% Only for the journal Data:
%\dataset{DOI number or link to the deposited data set in cases where the data set is published or set to be published separately. If the data set is submitted and will be published as a supplement to this paper in the journal Data, this field will be filled by the editors of the journal. In this case, please make sure to submit the data set as a supplement when entering your manuscript into our manuscript editorial system.}

%\datasetlicense{license under which the data set is made available (CC0, CC-BY, CC-BY-SA, CC-BY-NC, etc.)}

%%%%%%%%%%%%%%%%%%%%%%%%%%%%%%%%%%%%%%%%%%
% Only for the journal Toxins
%\keycontribution{The breakthroughs or highlights of the manuscript. Authors can write one or two sentences to describe the most important part of the paper.}

%%%%%%%%%%%%%%%%%%%%%%%%%%%%%%%%%%%%%%%%%%
% Only for the journal Encyclopedia
%\encyclopediadef{Instead of the abstract}
%\entrylink{The Link to this entry published on the encyclopedia platform.}
%%%%%%%%%%%%%%%%%%%%%%%%%%%%%%%%%%%%%%%%%%
\begin{document}
%%%%%%%%%%%%%%%%%%%%%%%%%%%%%%%%%%%%%%%%%%

\section{Introduction}

%The introduction should briefly place the study in a broad context and highlight why it is important. It should define the purpose of the work and its significance. The current state of the research field should be reviewed carefully and key publications cited. Please highlight controversial and diverging hypotheses when necessary. Finally, briefly mention the main aim of the work and highlight the principal conclusions. As far as possible, please keep the introduction comprehensible to scientists outside your particular field of research. Citing a journal paper \cite{ref-journal}. Now citing a book reference \cite{ref-book1,ref-book2} or other reference types \cite{ref-unpublish,ref-communication,ref-proceeding}. Please use the command \citep{ref-thesis,ref-url} for the following MDPI journals, which use author--date citation: Administrative Sciences, Arts, Econometrics, Economies, Genealogy, Histories, Humanities, IJFS, Journal of Intelligence, Journalism and Media, JRFM, Languages, Laws, Religions, Risks, Social Sciences.

Успешное применение методов машинного обучения (ML) для решения широкого спектра задач приводит к созданию новых подходов к эффективному использованию идей ML в различных областях приложений. 

Примером класса задач, в которых методы машинного обучения продемонстрировали свою эффективность, являются задачи выявления основных свойств исследуемых явлений (например, физических, экономических или социальных), которые характеризуются стохастической природой или наличием скрытых параметров \cite{Golovenkin2020,Gonoskov2019}.

Методы машинного обучения успешно используются и для решения сложных задач вычислительной математики, например, для simulation of dynamical systems \cite{Seleznev2019}, решения ordinary, partial or stochastic differential equations \cite{Lagaris1998,Blechschmidt2021,Xu2020} .

Одной из таких сложных задач вычислительной математики, к решению которой можно так или иначе применить методы машинного обучения, являются задачи глобальной оптимизации. 
В таких задачах, как правило, не представляется возможным найти решение аналитически и возникает необходимость построения численных методов для его поиска.

Проблема численного решения задач оптимизации сопряжена со значительными трудностями. Во многом они связаны с размерностью и видом целевой функции. При этом наиболее сложными являются задачи, в которых целевая функция является многоэкстремальной, недифференцируемой и, более того, заданной в форме черного ящика (т. е. в виде некоторой вычислительной процедуры, на вход которой подается аргумент, а выходом является соответствующее значение функции). Методы решения именно таких задач рассматриваются в данной статье. 

Можно выделить несколько подходов к построению численных методов решения задач глобальной оптимизации. 

%Алгоритмы на основе метамоделей

Ряд алгоритмов основан на идее мультистарта: запуск локального поиска либо начиная с разных стартовых точек, либо с использованием разных параметров. Методы локальной оптимизации обладают большой скоростью сходимости. При этом одной из основных проблем в мультистартовых схемах является выбор начальных точек, которые бы соответствовали областям притяжения различных локальных решений. 
К решению данной проблемы могут быть успешно применены методы машинного обучения. 
Например, в \cite{RinnooyKan1987} для выбора перспективных стартовых точек использовались методы кластерного анализа. 
В \cite{Cassioli2012} область для запуска локального метода выделялась на основе классификации стартовых точке с помощью support vector machine.

Еще одним популярным классом методов, применимым к решению задач глобальной оптимизации, являются метаэвристические алгоритмы. 
Многие из них основаны на имитации процессов, протекающих в живой природе. Для настройки параметров таких алгоритмов также применимы методы машинного обучения. Например, в \cite{Jin2005} приведен обзор использования методов машинного обучения в эволюционных алгоритмах.

Следует отметить, что алгоритмы последних двух классов не обеспечивают гарантированную сходимость к решению задачи и проигрывают детерминированным алгоритмам по качеству работы \cite{Kvasov2018,Sergeyev2018} (e.g., measured by the number of correctly solved problems from a certain set). Поэтому перспективным является использование именно детерминированных методов.


One of the efficient deterministic methods for solving multiextremal optimization problems is \textit{the information-statistical global search algorithm} \cite{Strongin2000}. This method initially proposed for solving unconstrained optimization problems was successfully generalized to the classes of optimization problems with non-convex constraints and multicriteria optimization problems. For different versions of the algorithm, parallelization methods taking into account the architecture of modern computing systems were also suggested \cite{Barkalov2016,globalizerSystem,Strongin2018}.


Для алгоритма глобального поиска в разное время было предложено несколько стратегий ускорения его работы (в терминах числа итераций, требующихся для решения задачи с заданной точностью). В данной работе предлагается новый подход к ускорению, основанный на выявления областей притяжения локальных минимумов с использованием методов машинного обучения. Выявление областей притяжения и запуск в этих областях локального поиска позволяет существенным образом сократить число испытаний, требующееся методу для достижения глобальной сходимости. Сказанное подтверждается результатами экспериментов, проведенных на серии из нескольких сотен тестовых задач. 



\section{Problem Statement}

Будем рассматривать задачи глобальной оптимизации вида
\begin{eqnarray}\label{main_problem}
& \varphi(y^\ast)=\min{\left\{\varphi(y):y\in D\right\}},\\
& D=\left\{y\in \text{R}^N: a_i\leq y_i \leq b_i, 1\leq i \leq N\right\}. \nonumber
\end{eqnarray}
Задача \ref{main_problem} рассматривается в предположении, что целевая функция является многоэкстремальной, задана как ``черный ящик'', а вычисление ее значений связано с решением задачи численного моделирования и является трудоемкой операцией.

Для многих прикладных задач типичной является ситуация, когда ограниченное изменение вектора параметров $y$ вызывает ограниченное изменение значений $\varphi(y)$. Математической моделью, описывающей указанное предположение, является предположение о выполнимости условия Липшица
\[
\left|\varphi(y')-\varphi(y'')\right|\leq L\left\|y'-y''\right\|,\; y',y'' \in D,\; 0<L<\infty.
\]
%Отметим, что значение константы $L$ обычно является априори неизвестным, что делает ее оценку одной из ключевых проблем при построении методов липшицевой оптимизации.$\overline{\underline{\text{XX}}}$
Предположение липшицевости целевой функции типично для многих подходов к разработке оптимизационных алгоритмов \cite{Jones1993,Pinter1996,Zilinskas2008,Evtushenko2009}.
При этом многие известные методы основаны на различных способах разбиения области поиска на систему подобластей и последующего выбора наиболее перспективной подобласти для размещения очередного испытания (вычисления значения целевой функции) \cite{Jones2009,Zilinskas2010,Evtushenko2013,Kvasov2013,Paulavicius2016}. 

Важным свойством задач глобальной оптимизации является тот факт, что в отличие от задач на поиск локального экстремума, глобальный минимум является интегральной характеристикой решаемой задачи. Чтобы убедиться, что точка $y^*\in D$ является решением задачи, недостаточно исследовать лишь ее окрестность, требуется исследование всей области поиска. Как результат, при минимизации существенно многоэкстремальных функций численный метод должен построить покрытие области поиска и число узлов этого покрытия увеличивается экспоненциально с ростом размерности. 
Эта особенность обуславливает высокую трудоемкость решения задач многоэкстремальной оптимимизации, и размерность является решающим фактором, влияющим на сложность их решения. 

Для преодоления сложностей, вызванных размерностью решаемой задачи, в многоэкстремальной оптимизации широко используются различные подходы к уменьшению размерности.  
Например, симплексное или диагональное разбиение области поиска позволяет использовать для решения исходной многомерной задачи методы решения одномерных задач (see, for example, \cite{PaulaviciusZilinskas2014,Sergeyev2017,Sergeyev2013}). 
Известным подходом к редукции размерости является также использование Peano space-filling curves mapping the multidimensional domain onto one-dimensional interval \cite{Strongin2000}.

В данной работе мы будем использовать еще один способ, основанный на the nested optimization scheme \cite{Shi2000,Grishagin2001,VanDam2010,Grishagin2015} and in its generalization \cite{Grishagin2016,Grishagin2016_1}.
The nested optimization scheme, с одной стороны, не ухудшает свойств минимизируемой функции (в отличие от редукции с использованием Peano curves), а с другой стороны, не требует использования сложных структур данных для поддержки симплексных или диагональных разбиений допустимой области. 
Одновременно с этим схема вложенной оптимизации позволяет свести решение многомерной задачи к решению набора стандартных одномерных задач и использовать широкий спектр алгоритмов одномерной глобальной оптимизации для их решения \cite{}.

\section{Methods}

\subsection{Core global search algorithm}\label{CoreGSA}

В качестве стандартной задачи рассмотрим одномерную задачу многоэкстремальной оптимизации 
\begin{equation}\label{uni_problem}
\varphi^\ast = \varphi(x^\ast)=\min{\left\{\varphi(x):x\in \left[0,1\right] 
\right\}}
\end{equation}
с целевой функцией, удовлетворяющей условию Липшица.

Приведем описание алгоритма глобального поиска (АГП) для решения базовой задачи в соответствии с \cite{Strongin2000}.
В процессе своей работы АГП порождает последовательность точек $x^i$, в которых вычисляются значения целевой функции $z^i=\varphi(x^i)$. 
Будем называть процесс вычисления значения целевой функции \textit{испытанием}.

В соответствии с алгоритмом первые два испытания проводятся в граничных точках отрезка $[0,1]$, т.е. $x^0=0,\;x^1=1$. 
В этих точках вычисляются значения целевой функции $z^0=\varphi(x^0),\;z^1=\varphi(x^1)$ и устанавливается значение счетчика $k=1$. 
Точка очередного испытания $x^{k+1}, k\geq 1,$ выбирается в соответствии со следующими действиями.

 Step 1. Перенумеровать нижним индексом (начиная с 0) точки $x^i,\:0\leq i\leq k$, проведенных испытаний в порядке возрастания координаты, т.е.
\[
0=x_0<x_1<\ldots <x_{k}=1.
\] 
Сопоставить точкам $x_i, \; 0\leq i\leq k$, вычисленные в них значения целевой функции $z_i=\varphi(x_i), \; 0\leq i\leq k$.

Step 2. Вычислить максимальное абсолютное значение относительной первой разности
\begin{equation}\label{mu}
\mu=\max_{1\leq i\leq k}\frac{\left|z_i-z_{i-1}\right|}{\Delta_i},
\end{equation}
где $\Delta_i = x_i-x_{i-1}$. Если вычисленное в соответствии с данной формулой значение равно нулю, то положить $\mu = 1$.

Step 3. Для всех интервалов $(x_{i-1},x_i),1\leq i\leq k$,  вычислить значение
\begin{equation}\label{R}
R(i)=r\mu\Delta_i+\frac{(z_i-z_{i-1})^2}{r\mu\Delta_i}-2(z_i+z_{i-1}),
\end{equation} 
называемое \textit{характеристикой} интервала; величина $r>1$ является параметром алгоритма. 

Step 4. Найти интервал $(x_{t-1},x_t)$ с максимальной характеристикой
\begin{equation}\label{MaxR}
R(t)=\max_{1\leq i\leq {k}}R(i).
\end{equation}
Если максимальная характеристика соответствует нескольким интервалам, то в качестве $t$ выбрать минимальное число, удовлетворяющее (\ref{MaxR}).

Step 5. Провести новое испытание в точке
\begin{equation}\label{xk1}
x^{k+1}=\frac{1}{2}(x_{t-1}+x_t) - \frac{z_t-z_{t-1}}{2r\mu}.
\end{equation}

Алгоритм прекращает свою работу при выполнении условия $\Delta_t<\epsilon$; здесь $t$ из (\ref{MaxR}), а $\epsilon>0$ есть заданная точность. 
В качестве оценки решения задачи выбираются значения 
\[
z_k^\ast=\min_{0\leq i \leq k}\varphi(x^i), \ x_k^\ast=\arg \min_{0\leq i \leq
 k}\varphi(x^i).
\] 

Теоретические условия, определяющие сходимость алгоритма, представлены в \cite{Strongin2000}.

%Пример работы алгоритма - функция Хилла. Эту же задачу надо будет показать и в след.разделе


\subsection{Machine Learning Regression как инструмент для выделения областей притяжения локальных экстремумов}

Функции, рассматриваемые в рамках данного исследования, принадлежат классу липшицевых функций. Поэтому классические методы регрессии (например, полиномиальная регрессия, когда функция приближается полиномом заданной степени) не будут хорошо соответствовать поведению функции. 

Более мощным инструментом являются регрессионные сплайны. При построении регрессионного сплайна область определения разбивается на $K$ непересекающихся подобластей, в каждой из которых функция аппроксимируется полиномом. При разбиении интервала на достаточное число подобластей позволяет очень точно аппроксимировать исходную функцию. 

Построить регрессию можно также и с использованием такого мощного инструмента, как искусственные нейронные сети. 
Для построения регрессии могут быть использованы сети разного типа, например, многослойный персептрон, сеть радиальных базисных функций, и т.д. 
Однако в обоих этих случаях модель (сплайн или же нейронная сеть) сама становится сложной для того анализа, который требуется выполнить в решаемой проблеме (выделение областей притяжения локальных экстремумов). 

Поэтому в рамках проведенного исследования для анализа локального поведения функции мы выбрали регрессионную модель, основанную на деревьях решений. 
Например, если целевая функция хорошо аппроксимируется полиномами, то полиномиальная регрессия, конечно же, будет адекватно передавать свойства функции. Однако если между имеет место более сложная зависимость, тогда дерево решений может превзойти по качеству аппроксимации классические варианты регрессии.
%Здесь нужен рисунок - функция Шекеля, и два варианта регрессии - полином и деревья решений. Шекель должен плохо полиномами представляться, т.к. там функции вида 1/x
Одновременно с этим регрессия на основе дерева решений позволяет с достаточной точностью легко выделять области притяжения локальных экстремумов.

Построение регрессии с помощью дерева решений включает два основных этапа:
\begin{itemize}
	\item Область поиска $D$ разбивается на $J$ непересекающихся подобластей $D_1, D_2, ..., D_J$, причем $D = \bigcup_{j=1}^{J}{D_j}$.
	\item Любому значению, попадающему в область $D_j$, т.е. $x \in D_j$, ставится в соответствие среднее значение $c_j$ по значениям training trials, попавших в эту область.
\end{itemize}

Фактически, деревья решений представляют модель функции вида
\begin{equation}\label{dtree}
f(x) = c_j, \; x\in D_j.
\end{equation}
 
Указанная модель является, с одной стороны, довольно простой, а с другой стороны, адекватно отражает интересующие нас свойства функции (наличие или отсутствие локальных минимумов). Выделение областей притяжения локальных минимумов с использованием модели (\ref{dtree}) можно организовать следующим образом.
%Здесь - описание алгоритма выделения локальных минимумов, с примером на Шекеле.


\subsection{Адаптивная схема редукции размерности }

Рекурсивная схема вложенной оптимизации основана на известном соотношении \cite{Grishagin2001} 
\begin{equation}\label{nested}
\min_{y \in D}\varphi(y) = \min_{y_1\in\left[a_1,b_1\right]}\min_{y_2\in\left[a_2,b_2\right]}...\min_{y_N\in\left[a_N,b_N\right]}\varphi(y),
\end{equation}
которое позволяет свести решение исходной многомерной задачи (\ref{main_problem}) к решению семейства рекурсивно связанных одномерных подзадач.

Для формального описания многошаговой схемы введем семейство функций, определяемых в соответствии с соотношениями 
\begin{equation}\label{nested_N}
\varphi_N(y_1,...,y_N) \equiv \varphi(y_1,...,y_N),
\end{equation}
\begin{equation}\label{nested_i}
\varphi_i(y_1,...,y_i) = \min_{ y_{i+1} \in\left[a_{i+1},b_{i+1}\right]} \varphi_{i+1}(y_1,...,y_i,y_{i+1}), 1\leq i\leq N-1.
\end{equation}

Тогда, в соответствии с (\ref{nested}), для решения многомерной задачи (\ref{main_problem}) достаточно решить одномерную задачу  
\begin{equation}\label{nested_1}
\varphi^* = \min_{y_1\in\left[a_1,b_1\right]}\varphi_1(y_1).
\end{equation}
Однако каждое вычисление значения функции $\varphi_1$ в некоторой фиксированной точке $y_1$ предполагает решение одномерной задачи оптимизации второго уровня 
\begin{equation}
\varphi_1(y_1) = \min_{y_2\in\left[a_2,b_2\right]}\varphi_2(y_1,y_2).
\end{equation}
Вычисление значений функции $\varphi_2$ в свою очередь требует одномерной минимизации функции $\varphi_3$ и т.д. вплоть до решения задачи
\begin{equation}
\varphi_{N-1}(y_1,...,y_{N-1}) = \min_{ y_{N} \in\left[a_{N},b_{N}\right]} \varphi_{N}(y_1,...,y_{N})
\end{equation}
на последнем уровне рекурсии.

Решение возникающего в схеме вложенной оптимизации множества подзадач (\ref{nested_i}) может быть организовано различными способами. 
Очевидный способ (детально проработанный в \cite{Grishagin2001,Grishagin2015} основан на решении подзадач в соответствии с рекурсивным порядком их порождения. Однако здесь возникает потеря значительной части информации о целевой функции. 

Иным  подходом является адаптивная схема, в которой все подзадачи решаются одновременно, что позволяет более полно учитывать информацию о многомерной задаче и за счет этого ускорять процесс ее решения.
Данный подход был теоретически обоснован и апробирован в \cite{Grishagin2016,Grishagin2016_1,Grishagin2018}. 

Кратко отметим, что в рамках исходной схемы вложенной оптимизации порождаемые подзадачи решаются строго последовательно; получаемая в результате иерархическая схема порождения и решения подзадач имеет вид дерева. Построение этого дерева происходит динамически в процессе решения исходной задачи (\ref{main_problem}). При этом вычисление одного значения функции $\varphi_i(y_1,y_2,...,y_i)$ на $i$-м уровне требует полного решения всех задач одного из поддеревьев уровня $i+1$.

Адаптивная многошаговая схема редукции размерности изменяет порядок решения подзадач: они будут решаться не по одной (в соответствии с их иерархией в дереве задач), а одновременно, т.е. будет существовать некоторое множество подзадач, находящихся в процессе решения. В рамках адаптивной схемы:
\begin{itemize}
	\item 
для вычисления значения функции $i$-го уровня из (\ref{nested_i}) порождается новая задача уровня $i+1$, в которой проводится только одно испытание, после чего новая порожденная задача включается в множество уже имеющихся задач, подлежащих решению;
	\item 
	итерация глобального поиска состоит в выборе одной (наиболее перспективной) задачи из множества имеющихся задач, в которой проводится одно испытание; точка проведения нового испытания определяется в соответствии с базовым алгоритмом глобального поиска из subsection \ref{CoreGSA};
	\item
в качестве минимальных значений функций из (\ref{nested_i}) используются их текущие оценки, полученные на основе накопленной поисковой информации.
\end{itemize}

Краткое описание основных шагов адаптивной схемы редукции размерности состоит в следующем.

Пусть вложенные подзадачи вида (\ref{nested_i}) решаются с помощью алгоритма глобального поиска, описанного в subsection \ref{CoreGSA}. Тогда каждой подзадаче (\ref{nested_i}) можно присвоить некоторое числовое значение, называемое характеристикой этой задачи. В качестве такой характеристики можно взять значение $R(t)$ из (\ref{MaxR}), т.е. максимальную характеристику из характеристик интервалов, сформированных в данной задаче. Чем выше значение данной характеристики, тем более перспективной является подзадача для продолжения поиска в ней глобального минимума исходной задачи (\ref{main_problem}). Поэтому на каждой итерации выбирается подзадача с максимальной характеристикой для проведения в ней очередного испытания. Это испытание либо приводит к вычислению значения целевой функции $\varphi(y)$ (если выбранная подзадача принадлежала уровню $j=M$), либо порождает новые подзадачи согласно (\ref{nested_i}) при $j\leq M-1$. В последнем случае новые порожденные задачи добавляются к текущему множеству задач, вычисляются их характеристики и процесс повторяется. Завершение процесса оптимизации происходит, когда в корневой задаче выполняется условие остановки алгоритма, решающего эту задачу.



\section{Experimental Results}



\section{Conclusions}
 
%%%%%%%%%%%%%%%%%%%%%%%%%%%%%%%%%%%%%%%%%%
\section{Materials and Methods}

Materials and Methods should be described with sufficient details to allow others to replicate and build on published results. Please note that publication of your manuscript implicates that you must make all materials, data, computer code, and protocols associated with the publication available to readers. Please disclose at the submission stage any restrictions on the availability of materials or information. New methods and protocols should be described in detail while well-established methods can be briefly described and appropriately cited.

Research manuscripts reporting large datasets that are deposited in a publicly avail-able database should specify where the data have been deposited and provide the relevant accession numbers. If the accession numbers have not yet been obtained at the time of submission, please state that they will be provided during review. They must be provided prior to publication.

Interventionary studies involving animals or humans, and other studies require ethical approval must list the authority that provided approval and the corresponding ethical approval code.
\begin{quote}
This is an example of a quote.
\end{quote}

%%%%%%%%%%%%%%%%%%%%%%%%%%%%%%%%%%%%%%%%%%
\section{Results}

This section may be divided by subheadings. It should provide a concise and precise description of the experimental results, their interpretation as well as the experimental conclusions that can be drawn.
\subsection{Subsection}
\subsubsection{Subsubsection}

Bulleted lists look like this:
\begin{itemize}
\item	First bullet;
\item	Second bullet;
\item	Third bullet.
\end{itemize}

Numbered lists can be added as follows:
\begin{enumerate}
\item	First item; 
\item	Second item;
\item	Third item.
\end{enumerate}

The text continues here. 

\subsection{Figures, Tables and Schemes}

All figures and tables should be cited in the main text as Figure~\ref{fig1}, Table~\ref{tab1}, etc.

%\begin{figure}[H]
%\includegraphics[width=10.5 cm]{Definitions/logo-mdpi}
%\caption{This is a figure. Schemes follow the same formatting. If there are multiple panels, they should be listed as: (\textbf{a}) Description of what is contained in the first panel. (\textbf{b}) Description of what is contained in the second panel. Figures should be placed in the main text near to the first time they are cited. A caption on a single line should be centered.\label{fig1}}
%\end{figure}   

% The MDPI table float is called specialtable
\begin{specialtable}[H] 
\caption{This is a table caption. Tables should be placed in the main text near to the first time they are~cited.\label{tab1}}
%%% \tablesize{} %% You can specify the fontsize here, e.g., \tablesize{\footnotesize}. If commented out \small will be used.
\begin{tabular}{ccc}
\toprule
\textbf{Title 1}	& \textbf{Title 2}	& \textbf{Title 3}\\
\midrule
Entry 1		& Data			& Data\\
Entry 2		& Data			& Data\\
\bottomrule
\end{tabular}
\end{specialtable}

%\begin{listing}[H]
%\caption{Title of the listing}
%\rule{\columnwidth}{1pt}
%\raggedright Text of the listing. In font size footnotesize, small, or normalsize. Preferred format: left aligned and single spaced. Preferred border format: top border line and bottom border line.
%\rule{\columnwidth}{1pt}
%\end{listing}

Text.

Text.

\subsection{Formatting of Mathematical Components}

This is the example 1 of equation:
\begin{equation}
a = 1,
\end{equation}
the text following an equation need not be a new paragraph. Please punctuate equations as regular text.
%% If the documentclass option "submit" is chosen, please insert a blank line before and after any math environment (equation and eqnarray environments). This ensures correct linenumbering. The blank line should be removed when the documentclass option is changed to "accept" because the text following an equation should not be a new paragraph.

This is the example 2 of equation:
\end{paracol}
\nointerlineskip
\begin{equation}
a = b + c + d + e + f + g + h + i + j + k + l + m + n + o + p + q + r + s + t + u + v + w + x + y + z
\end{equation}

% Example of a figure that spans the whole page width (the commands \widefigure and \begin{paracol}{2}, \linenumbers, and\switchcolumn need to be present). The same concept works for tables, too.
%\begin{figure}[H]	
%\widefigure
%\includegraphics[width=15 cm]{Definitions/logo-mdpi}
%\caption{This is a wide figure.\label{fig2}}
%\end{figure}  
\begin{paracol}{2}
\linenumbers
\switchcolumn

Please punctuate equations as regular text. Theorem-type environments (including propositions, lemmas, corollaries etc.) can be formatted as follows:
%% Example of a theorem:
\begin{Theorem}
Example text of a theorem.
\end{Theorem}

The text continues here. Proofs must be formatted as follows:

%% Example of a proof:
\begin{proof}[Proof of Theorem 1]
Text of the proof. Note that the phrase ``of Theorem 1'' is optional if it is clear which theorem is being referred to.
\end{proof}
The text continues here.

%%%%%%%%%%%%%%%%%%%%%%%%%%%%%%%%%%%%%%%%%%
\section{Discussion}

Authors should discuss the results and how they can be interpreted from the perspective of previous studies and of the working hypotheses. The findings and their implications should be discussed in the broadest context possible. Future research directions may also be highlighted.

%%%%%%%%%%%%%%%%%%%%%%%%%%%%%%%%%%%%%%%%%%
\section{Conclusions}

This section is not mandatory, but can be added to the manuscript if the discussion is unusually long or complex.

%%%%%%%%%%%%%%%%%%%%%%%%%%%%%%%%%%%%%%%%%%
%\section{Patents}

%This section is not mandatory, but may be added if there are patents resulting from the work reported in this manuscript.

%%%%%%%%%%%%%%%%%%%%%%%%%%%%%%%%%%%%%%%%%%
\vspace{6pt} 

%%%%%%%%%%%%%%%%%%%%%%%%%%%%%%%%%%%%%%%%%%
%% optional
%\supplementary{The following are available online at \linksupplementary{s1}, Figure S1: title, Table S1: title, Video S1: title.}

% Only for the journal Methods and Protocols:
% If you wish to submit a video article, please do so with any other supplementary material.
% \supplementary{The following are available at \linksupplementary{s1}, Figure S1: title, Table S1: title, Video S1: title. A supporting video article is available at doi: link.} 

%%%%%%%%%%%%%%%%%%%%%%%%%%%%%%%%%%%%%%%%%%
\authorcontributions{For research articles with several authors, a short paragraph specifying their individual contributions must be provided. The following statements should be used ``Conceptualization, X.X. and Y.Y.; methodology, X.X.; software, X.X.; validation, X.X., Y.Y. and Z.Z.; formal analysis, X.X.; investigation, X.X.; resources, X.X.; data curation, X.X.; writing---original draft preparation, X.X.; writing---review and editing, X.X.; visualization, X.X.; supervision, X.X.; project administration, X.X.; funding acquisition, Y.Y. All authors have read and agreed to the published version of the manuscript.'', please turn to the  \href{http://img.mdpi.org/data/contributor-role-instruction.pdf}{CRediT taxonomy} for the term explanation. Authorship must be limited to those who have contributed substantially to the work~reported.}

\funding{This research was funded by the Ministry of Science and Higher Education of the Russian Federation, agreement number 075-15-2020-808.}

\institutionalreview{Not applicable.}

\informedconsent{Not applicable.}

%\dataavailability{In this section, please provide details regarding where data supporting reported results can be found, including links to publicly archived datasets analyzed or generated during the study. Please refer to suggested Data Availability Statements in section ``MDPI Research Data Policies'' at \url{https://www.mdpi.com/ethics}. You might choose to exclude this statement if the study did not report any data.} 

%\acknowledgments{In this section you can acknowledge any support given which is not covered by the author contribution or funding sections. This may include administrative and technical support, or donations in kind (e.g., materials used for experiments).}

\conflictsofinterest{The authors declare no conflict of interest.} 

%% Optional
%\sampleavailability{Samples of the compounds ... are available from the authors.}

%%%%%%%%%%%%%%%%%%%%%%%%%%%%%%%%%%%%%%%%%%
%% Only for journal Encyclopedia
%\entrylink{The Link to this entry published on the encyclopedia platform.}

%%%%%%%%%%%%%%%%%%%%%%%%%%%%%%%%%%%%%%%%%%
%% Optional
%\abbreviations{Abbreviations}{
%The following abbreviations are used in this manuscript:\\
%
%\noindent 
%\begin{tabular}{@{}ll}
%MDPI & Multidisciplinary Digital Publishing Institute\\
%DOAJ & Directory of open access journals\\
%TLA & Three letter acronym\\
%LD & Linear dichroism
%\end{tabular}}
%
%%%%%%%%%%%%%%%%%%%%%%%%%%%%%%%%%%%%%%%%%%%
%%% Optional
%\appendixtitles{no} % Leave argument "no" if all appendix headings stay EMPTY (then no dot is printed after "Appendix A"). If the appendix sections contain a heading then change the argument to "yes".
%\appendixstart
%\appendix
%\section{}
%\subsection{}
%The appendix is an optional section that can contain details and data supplemental to the main text---for example, explanations of experimental details that would disrupt the flow of the main text but nonetheless remain crucial to understanding and reproducing the research shown; figures of replicates for experiments of which representative data are shown in the main text can be added here if brief, or as Supplementary Data. Mathematical proofs of results not central to the paper can be added as an appendix.
%
%\begin{specialtable}[H] 
%%\tablesize{\scriptsize}
%\caption{This is a table caption. Tables should be placed in the main text near to the first time they are~cited.\label{tab2}}
%%\tablesize{} % You can specify the fontsize here, e.g., \tablesize{\footnotesize}. If commented out \small will be used.
%\begin{tabular}{ccc}
%\toprule
%\textbf{Title 1}	& \textbf{Title 2}	& \textbf{Title 3}\\
%\midrule
%Entry 1		& Data			& Data\\
%Entry 2		& Data			& Data\\
%\bottomrule
%\end{tabular}
%\end{specialtable}
%
%\section{}
%All appendix sections must be cited in the main text. In the appendices, Figures, Tables, etc. should be labeled, starting with ``A''---e.g., Figure A1, Figure A2, etc. 

%%%%%%%%%%%%%%%%%%%%%%%%%%%%%%%%%%%%%%%%%%
\end{paracol}
%%%%%%%%%%%%%%%%%%%%%%%%%%%%%%%%%%%%%%%%%%
\reftitle{References}

% Please provide either the correct journal abbreviation (e.g. according to the “List of Title Word Abbreviations” http://www.issn.org/services/online-services/access-to-the-ltwa/) or the full name of the journal.
% Citations and References in Supplementary files are permitted provided that they also appear in the reference list here. 

%=====================================
% References, variant A: external bibliography
%=====================================
\externalbibliography{yes}
\bibliography{bibliography}

%=====================================
% References, variant B: internal bibliography
%=====================================
%\begin{thebibliography}{999}
%% Reference 1
%\bibitem[Author1(year)]{ref-journal}
%Author~1, T. The title of the cited article. {\em Journal Abbreviation} {\bf 2008}, {\em 10}, 142--149.
%% Reference 2
%\bibitem[Author2(year)]{ref-book1}
%Author~2, L. The title of the cited contribution. In {\em The Book Title}; Editor1, F., Editor2, A., Eds.; Publishing House: City, Country, 2007; pp. 32--58.
%% Reference 3
%\bibitem[Author3(year)]{ref-book2}
%Author 1, A.; Author 2, B. \textit{Book Title}, 3rd ed.; Publisher: Publisher Location, Country, 2008; pp. 154--196.
%% Reference 4
%\bibitem[Author4(year)]{ref-unpublish}
%Author 1, A.B.; Author 2, C. Title of Unpublished Work. \textit{Abbreviated Journal Name} stage of publication (under review; accepted; in~press).
%% Reference 5
%\bibitem[Author5(year)]{ref-communication}
%Author 1, A.B. (University, City, State, Country); Author 2, C. (Institute, City, State, Country). Personal communication, 2012.
%% Reference 6
%\bibitem[Author6(year)]{ref-proceeding}
%Author 1, A.B.; Author 2, C.D.; Author 3, E.F. Title of Presentation. In Title of the Collected Work (if available), Proceedings of the Name of the Conference, Location of Conference, Country, Date of Conference; Editor 1, Editor 2, Eds. (if available); Publisher: City, Country, Year (if available); Abstract Number (optional), Pagination (optional).
%% Reference 7
%\bibitem[Author7(year)]{ref-thesis}
%Author 1, A.B. Title of Thesis. Level of Thesis, Degree-Granting University, Location of University, Date of Completion.
%% Reference 8
%\bibitem[Author8(year)]{ref-url}
%Title of Site. Available online: URL (accessed on Day Month Year).
%\end{thebibliography}

% If authors have biography, please use the format below
%\section*{Short Biography of Authors}
%\bio
%{\raisebox{-0.35cm}{\includegraphics[width=3.5cm,height=5.3cm,clip,keepaspectratio]{Definitions/author1.pdf}}}
%{\textbf{Firstname Lastname} Biography of first author}
%
%\bio
%{\raisebox{-0.35cm}{\includegraphics[width=3.5cm,height=5.3cm,clip,keepaspectratio]{Definitions/author2.jpg}}}
%{\textbf{Firstname Lastname} Biography of second author}

% The following MDPI journals use author-date citation: Arts, Econometrics, Economies, Genealogy, Humanities, IJFS, JRFM, Laws, Religions, Risks, Social Sciences. For those journals, please follow the formatting guidelines on http://www.mdpi.com/authors/references
% To cite two works by the same author: \citeauthor{ref-journal-1a} (\citeyear{ref-journal-1a}, \citeyear{ref-journal-1b}). This produces: Whittaker (1967, 1975)
% To cite two works by the same author with specific pages: \citeauthor{ref-journal-3a} (\citeyear{ref-journal-3a}, p. 328; \citeyear{ref-journal-3b}, p.475). This produces: Wong (1999, p. 328; 2000, p. 475)

%%%%%%%%%%%%%%%%%%%%%%%%%%%%%%%%%%%%%%%%%%
%% for journal Sci
%\reviewreports{\\
%Reviewer 1 comments and authors’ response\\
%Reviewer 2 comments and authors’ response\\
%Reviewer 3 comments and authors’ response
%}
%%%%%%%%%%%%%%%%%%%%%%%%%%%%%%%%%%%%%%%%%%
\end{document}

