\documentclass{llncs}


\usepackage{hyperref}
\usepackage{graphicx}
\usepackage{multirow}
\usepackage[misc,geometry]{ifsym}

\usepackage{amsmath}
\usepackage{enumitem}
%\usepackage{amsfonts}
%\usepackage{amssymb}
%\usepackage{epstopdf}
%\usepackage{epsfig}
\usepackage[T2A]{fontenc}
\usepackage[utf8x]{inputenc}
\usepackage[russian,english]{babel}

\begin{document}

\mainmatter 

\title{Mixed-integer}
\author{Victor Gergel \and Konstantin Barkalov \and Ilya Lebedev %\Letter 
\\
\email{konstantin.barkalov@itmm.unn.ru}}

\institute{
Lobachevsky State University of Nizhni Novgorod, Nizhni Novgorod, Russia
}

\maketitle

\begin{abstract}
The paper presents .

\keywords global optimization, non-convex constraints, mixed-integer problems.

\end{abstract}

\section{Introduction}\label{sec:intro}
\Russian
В работе рассматриваются mixed-integer global optimization problems и методы их решения. Задачи глобальной оптимизации обладают значительной вычислительной трудоемкостью, т.к. глобальный оптимум является интегральной характеристикой решаемой задачи и требует исследования всей области поиска. Как результат, поиск глобального оптимума сводится к построению некоторого покрытия (вообще говоря, неравномерного) в области параметров. Особый интерес представляют задачи, в которых часть параметров может принимать лишь целочисленные значения, т.к. для них сложнее построить оценки оптимума по сравнению с непрерывными задачами.

Методам решения mixed-integer problems посвящена обширная литература (см., например, обзоры \cite{Burer,Boukouvala}). Известные детерминированные методы решения задач данного класса основаны, как правило, на идеях Branch-and-Bound or Branch-and-Reduce. Также известен ряд метаэвристических и генетических алгоритмов, так или иначе основанных на идеях случайного поиска.

В данной работе нами предложен новый детерминированный метод решения mixed-integer problems, основанный на information-statistical approach к решению задач глобальной оптимизации \cite{Strongin2000,Strongin2013}. Текст статьи, отражающей предварительные результаты исследования, построен следующим образом. Вначале рассмотрим подход к решению задач с непрерывными параметрами, затем будет предложено его обобщение для mixed-integer problems. В финальном разделе приведены результаты сравнения предложенного метода с известными аналогами.

\section{Global search algorithm and dimension reduction}

Let us consider a multiextremal optimization problem in the form
\begin{gather}\label{problem}
\varphi(y^\ast)=\min{\left\{\varphi(y):y\in D, \; g_i(y)\leq 0, \; 1 \leq i \leq m\right\}},\\
D=\left\{y\in R^N: a_i\leq y_i \leq b_i, 1\leq i \leq N\right\}.
\end{gather}
Suppose, that the objective function $\varphi(y)$ (henceforth denoted by $g_{m+1}(y)$) and
the left-hand sides $g_i(y), \; 1\leq i \leq m$, of the constraints satisfy the Lipschitz condition
\[ 
\left|g_i(y')-g_i (y'')\right| \leq L_i \left\|y'-y'' \right\|, \; y',y''\in S, \; 1\leq i \leq m+1, 
\]
with constants $L_i, \; 1 \leq i \leq m+1$, respectively,  and may be multiextremal.

Using a continuous single-valued mapping $y(x)$  (Peano-type space-filling curve) of the interval $[0,1]$ onto $D$, a multidimensional problem (\ref{problem}) can be reduced to a one-dimensional problem
\begin{equation}\label{problem1}
\varphi(y(x^\ast))=\min \left\{\varphi(y(x)): x \in [0,1], \; g_i(y(x))\leq 0, \; 1 \leq i \leq m\right\}.
\end{equation}
The reduction of dimensionality matches the multidimensional problem with a Lipschitzian objective function and Lipschitzian constraints with a one-dimensional problem where the respective functions satisfy the uniform H\"older condition (see \cite{Strongin2000}).

An efficient global search algorithm (GSA) for solving one-dimensional constrained optimization problem (\ref{problem1}) is considered in \cite{Strongin2000}. A detailed description of various variants of this algorithm is presented in \cite{ }.

\section{Global search algorithm for mixed-integer problems}

\Russian
Рассмотрим теперь способ адаптации GSA для решения mixed-integer global optimization problems вида 

\begin{gather}\label{problem_i}
\min{\left\{ g_{m+1}(y):y\in D, \; g_i(y)\leq 0, \; 1 \leq i \leq m\right\}},\\
D=\left\{a_i\leq y_i \leq b_i, \; 1\leq i \leq N, \; y_j \in Z, \; j \in J, \; y_i \in R, \; i \notin J \right\}.
\end{gather}

На основе задачи (\ref{problem_i}) сформируем множество задач 
\begin{gather}\label{problem_is}
\min{\left\{ g_{m+1}^s(y):y\in D^s, \; g_i^s(y)\leq 0, \; 1 \leq i \leq m\right\}},\\
D^s=\left\{ a_i\leq y_i \leq b_i, \;  y_i \in R, \; i \notin J \right\}, \; s\in\{1,...,S\}, 
\end{gather}
каждая из которых соответствует исходной задаче (\ref{problem_i}) с фиксированным набором целочисленных параметров. Число задач $S$ будет соответствовать числу возможных комбинаций целочисленных параметров.

Используя схему редукции размерности с помощью развертки $y(x)$ на основе множества задач (\ref{ }) можно составить единую задачу
\begin{equation}\label{problem_is1}
\min \left\{g_{m+1}^s(y(x)): x \in [0,S], \; g_i^s(y^s(x)) \leq 0, \; 1 \leq i \leq m\right\},
\end{equation}
где отображение $y^s(x)$  формируется на основе исходной развертки $y(x)$ следующим образом
\[
y^s(x)=y(x-s), \; x\in[s-1,s],\; s\in\{1,...,S\}
\]
При этом функции $g_i^s(y^s(x))$ могут иметь разрывы в точках $s\in \{1,...,S-1\}$, поэтому данные точки рассматриваются как выколотые, значение objective function and constraints в них не определено.

Для иллюстрации на рис. 1 изображены графики функций, соответствующих задаче с одним непрерывным и одним бинарным параметром.


Применяя к решению задачи (\ref{problem_is1}) алгоритм глобального поиска, мы найдем решение задачи (\ref{problem_i}). При этом основная часть испытаний будет проведена в $s$-й подзадаче, решение которой соответствует решению исходной задачи (\ref{problem_i}). В остальных подзадачах будет проведена лишь незначительная часть испытаний, т.к. решения данных подзадач являются локально-оптимальными по отношению к решению $s$-й подзадачи. Сказанное подтверждается рис. 1, где штрихами обозначены точки испытаний, выполненных при решении данной задачи.

Таким образом, мы сформировали Mixed Integer Global Search Algorithm (MIGSA). Можно показать, что условия сходимости данного алгоритма будут следовать из условий сходимости его прототипа (GSA).

\section{Results of experiments}

\Russian
Проведем сравнение предложенного метода MIGSA с genetic algorithm for solving mixed-integer optimization problems, реализованным в Matlab Global Optimization Toolbox. В табл. 1 приведено число испытаний, потребовавшееся для решения известных тестовых mixed-integer problems данными методами. Для обоих методов была использована одинаковая точность поиска решения $10^{-2}$. Результаты экспериментов показывают превосходство метода MIGSA как по числу итераций, так и по времени работы.

\begin{table}
	\caption{Сравнение эффективности методов MIGSA и GA}
	\label{tab:1}
	\center
	\begin{tabular}{cccccc}
		\hline\noalign{\smallskip}
		Test problem & \multicolumn{2}{c}{ GA } & & \multicolumn{2}{c}{MIGSA} \\
		\noalign{\smallskip} \cline{2-3} \cline{5-6} \noalign{\smallskip}
		 & $k$ & $t$ & & $k$ & $t$  \\
		\noalign{\smallskip} \hline \noalign{\smallskip}
		 Problem 2 \cite{Floudas}&	2081 &	0.199 & &	417 &	0.04 \\
		 Problem 3 \cite{Floudas}& 	3851 &	0.353 & & 3324 &	0.107 \\
		 Problem 6 \cite{Floudas}&	2081 &	0.165 & &	118 &	0.001 \\
		 Problem 1 \cite{Deep}   &	2081 &	0.154 & &	66 &	0.0007 \\
		 Problem 2 \cite{Deep}   &	2081 &	0.191 & &	57 &	0.0006 \\
		 Problem 7 \cite{Deep}   &	2321 &	0.284 & & 372	 &	0.017 \\
		\noalign{\smallskip}\hline
	\end{tabular}
\end{table}

Для демонстрации надежности метода MIGSA нами были решены четыре серии по 100 многоэкстремальных mixed-integer problems, построенных на основе модифицированных классов задач Simple и Hard, сгенерированных генератором GKLS [Sergeyev]. В задачах было 5 целочисленных и 3 непрерывных параметра. Точность поиска решения равнялась $10^{-2}$. Были решены все задачи из серии, при этом для решения задач на основе класса Simple потребовалось среднем 11988 trials, а для решения задач на основе класса Hard - 24750 trials.

\section*{Acknowledgment}
The authors are grateful for the support provided by the Russian Science Foundation, project No. 16-11-10150.


\begin{thebibliography}{10}

\bibitem{Burer}
Burer, S., Letchford, A.N.: Non-convex mixed-integer nonlinear programming: A survey. Surveys in Operations Research and Management Science \textbf{17}, 97--106 (2012) 

\bibitem{Boukouvala}
Boukouvala, F., Misener, R., Floudas, C.A.: Global optimization advances in Mixed-Integer Nonlinear Programming, MINLP, and Constrained Derivative-Free Optimization, CDFO. European Journal of Operational Research \textbf{252}, 701--727 (2016) 

\bibitem{Strongin2000}
Strongin, R.G., Sergeyev, Y.D.: Global Optimization with Non-Convex Constraints. Sequential and Parallel Algorithms. Kluwer Academic Publishers, Dordrecht (2000) %; DOI: 10.1007/978-1-4615-4677-1

\bibitem{Strongin2013}
Sergeyev, Ya.D., Strongin, R.G., Lera, D.: Introduction to global optimization exploiting space-filling curves. Springer (2013) %;  DOI: 10.1007/978-1-4614-8042-6

\bibitem{Floudas}
Floudas, C.A., Pardalos, P.M.:  Handbook of Test Problems in Local and Global Optimization. Springer (1999)  %; DOI: 10.1007/978-1-4757-3040-1

\bibitem{Deep}
Deep, K., Singh, K. P., Kansal, M.L., Mohan, C.: A real coded genetic algorithm for solving integer and mixed integer optimization problems. Appl. Math. Comput. \textbf{212}(2), 505--518 (2009)


\end{thebibliography}


\end{document}
______________________________________________________________________
