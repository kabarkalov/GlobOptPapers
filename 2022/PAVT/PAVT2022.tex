%%%%%%%%%%%%%%%%%%%% author.tex 
%%%%%%%%%%%%%%%%%%%%%%%%%%%%%%%%%%%
%
% sample root file for your "contribution" to a proceedings volume
%
% Use this file as a template for your own input.
%
%%%%%%%%%%%%%%%% Springer 
%%%%%%%%%%%%%%%%%%%%%%%%%%%%%%%%%%


\documentclass{svproc}
%
% RECOMMENDED 
%%%%%%%%%%%%%%%%%%%%%%%%%%%%%%%%%%%%%%%%%%%%%%%%
%%%
%
\usepackage{graphicx}
\usepackage{marvosym}
\usepackage{amsmath}
\usepackage{amssymb}
\usepackage{cite}
\usepackage{color}

%\usepackage[russian]{babel}

% to typeset URLs, URIs, and DOIs
\usepackage{url}
\usepackage{hyperref}
\def\UrlFont{\rmfamily}

\def\orcidID#1{\unskip$^{[#1]}$}
\def\letter{$^{\textrm{(\Letter)}}$}

\begin{document}
\mainmatter              % start of a contribution
% Решение обратных задач химической кинетки с помощью асинхронного алгоритма глобального поиска
% Решение обратных задач химической кинетки с помощью смешанного локально-глобального поискового алгоритма 
%\title{Solving the Inverse Problems of Chemical Kinetics Using the Asynchronous Global Optimization Algorithm}
\title{Kinetic Modeling of Isobutane Alkylation with Mixed C4 Olefins and Sulfuric Acid as a Catalyst Using the Asynchronous Global Optimization Algorithm}
\titlerunning{Kinetic Modeling of Isobutane Alkylation}  % abbreviated title (for running head)
%                                     also used for the TOC unless
%                                     \toctitle is used
%
\author{
Irek Gubaydullin$^{1,2}$\and
Leniza Enikeeva$^{2,3}$\orcidID{0000-0003-4219-4870}
\and
Konstantin Barkalov$^4$ \letter \orcidID{0000-0001-5273-2471}
\and
Ilya Lebedev$^4$\orcidID{0000-0002-8736-0652} 
\and
Dmitry Silenko$^4$ 
}

%
\authorrunning{I. Gubaydullin et al.} % abbreviated author list (for running head)
%
%%%% list of authors for the TOC (use if author list has to be modified)
%\tocauthor{Konstantin Barkalov and Ilya Lebedev }
%

\institute{$^1$Institute of Petrochemistry and Catalysis -- Subdivision of Ufa Federal Research Centre of the RAS, Ufa, Russia\\$^2$Ufa State Petroleum Technological University, Ufa, Russia \\$^3$Novosibirsk State University, Novosibirsk, Russia\\$^4$Lobachevsky State University of Nizhny Novgorod, Nizhny Novgorod, Russia\\
\email{leniza.enikeeva@yandex.ru},
\email{konstantin.barkalov@itmm.unn.ru},
\email{ilya.lebedev@itmm.unn.ru}
}

	
\maketitle              % typeset the title of the contribution

\begin{abstract}

The paper considers the application of parallel computing technology to the simulation of a catalytic chemical reaction, which is widely used in the modern automobile industry to produce gasoline with a high octane number. As a chemical reaction, the process of alkylation of isobutane with mixed C4 olefins, catalyzed by sulfuric acid, is assumed. To simulate a chemical process, it is necessary to develop a kinetic model of the process, i.e., to determine the kinetic parameters. To do this, the inverse problem of chemical kinetics is solved; it predicts the values of the kinetic parameters based on laboratory data. From a mathematical point of view, the inverse problem of chemical kinetics is a global optimization problem. A parallel asynchronous information-statistical global search algorithm was used to solve it. The use of the asynchronous algorithm significantly reduced the search time to find the optimum. The found optimal parameters of the model made it possible to adequately simulate the process of alkylation of isobutane with mixed C4 olefins catalyzed by sulfuric acid.

%обновить ключевые слова
\keywords{Global optimization $\cdot$ Multi-extremal functions $\cdot$ Parallel computing $\cdot$ Chemical kinetics $\cdot$ Inverse problems }
\end{abstract}

\section{Introduction}

%Часть УГНТУ
Currently, there is a tendency to improve the environmental characteristics of automobile fuel while maintaining a high octane number. Sulfuric acid alkylation of isobutane with olefins makes it possible to obtain a high-octane component of gasoline with a minimum content of aromatic hydrocarbons. The alkylate, which is produced by alkylation of isobutane with C3 -- C5 olefins in the presence of strong acid, has the advantages of a high octane number, low vapor pressure, and zero content of olefins and aromatics, making it a desirable blending component for high-quality gasoline. Alkylates will continue to act as a desirable blending component for high-quality gasoline as the quality of gasoline continues to increase \cite{cao2019}. Therefore, it is a significant process for a modern refinery. To optimize the chemical process in industry, it is necessary to develop first its model, which in this case means building a mathematical model of the chemical process, and then its kinetic model, i.e., to numerically calculate the kinetic constants of the reaction. 


%Стыковочный абзац
As a rule, it is impossible to find out the kinetic constants of reactions analytically. Therefore, there is a need in the development and application of numerical methods for finding the kinetic constants (see, e.g., \cite{Zaynullin2020,Enikeeva2020,Koledina2019,Enikeev2020,uskovfibrous,enikeeva2021,en13133393}). In this case, the quality criteria of the solution found (objective function) do not have an explicit analytical description, but enable an algorithmic representation and require considerable computational resources. Moreover, in the inverse problems of chemical kinetics, the objective function can be essentially multi-extremal, i.e., can have many local extrema along with the global one. 

%Часть ННГУ
Numerical methods for solving such multi-extremal problems (global optimization methods) differ significantly from local search methods (see, e.g., \cite{Sergeyev2017,PaulaviciusZilinskas2014}). As a rule, local optimization methods cannot escape the local extremum attraction region and do not find the global optimum. At the same time, the use of model parameters corresponding to the found local solution may appear to be insufficient since the global solution can provide a considerable advantage over local ones. 

The diversity of emerging global optimization problems entails various approaches to their solution. Methods for solving global optimization problems can be divided into two classes: metaheuristic and deterministic. Metaheuristic algorithms are usually based on the simulation of processes occurring in nature. Some examples of metaheuristic algorithms are simulated annealing, evolution and genetic algorithms, etc. (see, e.g., \cite{Battiti2009,Eiben2015}). Due to their relative simplicity, metaheuristic algorithms are more popular among researchers than deterministic methods. 
However, the problem solution found by the metaheuristic algorithm is, generally speaking, local and may be far from the global solution \cite{Kvasov2018}. 

The possibility to construct deterministic global search methods different from grid search and metaheuristic methods is related to the availability and consideration of some {\it a priori} assumption on the properties of problem functions. Such assumptions play a key role in the development of efficient global optimization algorithms and serve as the main mathematical tool for estimating global solutions.

The assumption on limited relative variations of objective function values is one of the natural assumptions of the problem. Such an assumption is related to the ratio of the function increment to the respective increment of its argument, which is usually limited by some threshold defined by the limited energy of variations in the simulated system. In this case, the functions are known as Lipschitz ones, and the problem itself is called the Lipschitz global optimization problem. 

This paper presents the results of applying parallel Lipschitz optimization methods for solving the inverse problems of chemical kinetics. The main part of the paper has the following structure. The description of the mathematical model of the investigated chemical reaction is presented in Section \ref{Sec_math_mod}. The formal statement of Lipschitz global optimization problems and the asynchronous parallel algorithm for solving them are described in Section \ref{Sec_GSA}. The results of the numerical solution of the inverse problem of chemical kinetics are discussed in Section \ref{Sec_Exp}.

\section{Problem Statement}\label{Sec_math_mod}
%Содержательная постановка задачи

Let us consider a mathematical model of the isobutane alkylation reaction with olefins in the presence of sulfuric acid, which is a system of ordinary nonlinear differential equations (\ref{eq:one})--(\ref{eq:twelve}).

\begin{gather}
  \frac{dc_1}{dt} = -k_1c_1 + k_2c_3 - k_3c_1c_3 - k_7c_1c_2c_4 - k_{11}c_1 + k_{14}c_{11} \label{eq:one} \\
  \frac{dc_2}{dt} = -k_4c_2c_4 - k_6c_2c_5 - k_7c_1c_2c_4 - k_{15}c_{11}c_2c_4 \label{eq:two} \\
  \frac{dc_3}{dt} = k_1c_1 + k_4c_2c_5 - k_3(c_1 + c_{11})c_3 - k_{5}c_{12}c_3 - k_2c_3 + k_7c_1c_2c_4 + k_{15}c_{11}c_2c_4 \label{eq:three} \\
  \frac{dc_4}{dt} = k_3(c_1 + k_{11})c_3 - k_4c_2c_4 - k_{7}c_{1}c_2c_4 - k_{15}c_{11}c_2c_4 \label{eq:four} \\
  \frac{dc_5}{dt} = k_5c_{12}c_3 - k_{6}c_2c_5 \label{eq:five} \\
  \frac{dc_6}{dt} = k_4c_{2}c_4 \label{eq:six} \\
  \frac{dc_7}{dt} = k_6c_{2}c_5 - k_{10}c_7 \label{eq:seven} \\
  \frac{dc_8}{dt} = k_7c_{1}c_2c_4 + k_{15}c_{11}c_2c_4 + k_9c_9c_{10} - k_8c_8 \label{eq:eight} \\
  \frac{dc_9}{dt} = k_8c_{8} - k_{9}c_{9}c_{10} \label{eq:nine} \\
  \frac{dc_{10}}{dt} = k_8c_{8} - k_{9}c_{9}c_{10} \label{eq:ten} \\
  \frac{dc_{11}}{dt} = -k_3c_{11}c_3 - k_{15}c_{11}c_{2}c_4 + k_{11}c_1 + k_{12}c_{12} - k_{13}c_{11} - k_{14}c_{11} \label{eq:eleven} \\
  \frac{dc_{12}}{dt} = -k_5c_{12}c_3 + k_{13}c_{11} - k_{12}c_{12} \label{eq:twelve} 
\end{gather}

The initial conditions are $t = 0, c_1 = c_1^0; c_2=c_2^0; c_3 = 0; c_4 = 0; c_5= 0; c_6 = 0; c_7 = 0; c_8 = 0; c_9 = 0; c_{10} = 0; c_{11}=c_{11}^0; c_{12} = c_{12}^0$.
The corresponding species in equations (\ref{eq:one})--(\ref{eq:twelve}) are 1, iC4H8; 2, iC4, 3, iC4+; 4, TMPs+; 5, DMHs+; 6, TMPs; 7, DMHs; 8, HEs; 9, iCx+; 10, iCy=; 11, 2-C4H8; 12, 1-C4H8.

Information that represents a change in the concentrations of the reaction components over time at different temperatures is taken as experimental data from \cite{cao2019}; the data in Table \ref{table1} are presented as an example of experimental data at a temperature of 276.2 K.

\begin{table}
\caption{Experimental data}
\label{table1}
\begin{center}
\begin{tabular}{ccccccc}
\hline
time, min & 1 & 2 & 5 & 10 & 15 & 20 \\
\hline\rule{0pt}{12pt}
DMH & 0.12 & 0.11 & 0.1	& 0.1 &	0.095 &	0.09  \\
TMP & 0.54 & 0.65 & 0.69 & 0.69 & 0.7 & 0.705 \\[2pt]
\hline
\end{tabular}
\end{center}
\end{table}

Thus, solving the system (\ref{eq:one})--(\ref{eq:twelve}) with the corresponding initial data, we will get a change in the calculated concentrations of the reaction components over time.

However, it is necessary to take into account the fact that the reaction rate constants $k_1, k_2, ..., k_{15}$ included in equations (\ref{eq:one})--(\ref{eq:twelve}) are parameters depending on the reaction temperature, this dependence is the Arrhenius equation and has the following form:

\begin{equation}
  k_i (T) = k_i^0 \exp \left(- \dfrac{E_i}{RT} \right),
  \label{eq:arren}
\end{equation}
where $k_i(T)$ is the constant of the $i$-th stage of the reaction rate, $k_i^0$ is the pre-exponential factor of the $i$-th reaction stage, $E_i$ is the activation energy, J/mol, $R$ is the universal gas constant, J $\cdot$ (K$\cdot$mol), $T$ is the temperature, K. Thus, to fully develop the kinetic model of the reaction, it is necessary to calculate the activation energies $E_i$ and the pre-exponential factors $k_i^0$ of all stages of the chemical reaction.  There are two formulations of the problems of searching for kinetic parameter data $E$ and $k^0$. The first one is to solve the inverse problem of selecting the kinetic parameters, included in equation (\ref{eq:arren}), which during the solution allow us to calculate all the reaction rate constants $k_i$. With the found rate constants of the reaction stages, the system of differential equations (\ref{eq:one})--(\ref{eq:twelve}) is solved, then the calculated concentrations are compared with the corresponding experimental data. Mathematically, this problem has the following formulation: it is necessary to minimize the following objective function

\begin{equation}
  F_1 = \sum_{i=1}^I \sum_{j=1}^J \sum_{k=1}^K \left| c_{ijk}^{exp} - c_{ijk}^{calc} \right| \longrightarrow \min,
  \label{eq:objective_func1}
\end{equation}
where $c_{ijk}^{exp}$ and $c_{ijk}^{calc}$ are the experimental and calculated values of the $k$-th observed component at the $i$-th experiment, respectively, $I$ is the number of observed temperatures for the reaction, ($I = 4$ in this case), $K$ is the number of experiments conducted at one temperature ($K = 3$), $M$ is the number of observed components of the reaction ($M = 2$).

The second formulation of the inverse problem of chemical kinetics implies the search for the rate constants of the stages $k_i$ included in the system (\ref{eq:one})--(\ref{eq:twelve}), separately for each temperature, then on the basis of Arrhenius equation~(\ref{eq:arren}), the activation energies $E_i$ and the pre-exponential multipliers $k_i^0$ are calculated using the least squares method. Mathematically, this statement of the problem coincides with equation (\ref{eq:objective_func1}), except for the first summation by temperatures:

\begin{equation}
  F_2 = \sum_{j=1}^J \sum_{k=1}^K \left| c_{ijk}^{exp} - c_{ijk}^{calc} \right| \longrightarrow \min.
  \label{eq:objective_func2} 
\end{equation}

Thus, both formulations of the problem (\ref{eq:objective_func1}) and (\ref{eq:objective_func2}) are optimization problems, and the next section will describe the method used to solve these minimization problems.

\section{Parallel Algorithm for Solving Global Optimization Problems }\label{Sec_GSA}

\subsection{Global Optimization Problem}

As mentioned above, from a formal point of view, we consider the inverse problem of chemical kinetics as a global optimization problem. 
In the specific problem under consideration, the values of the objective function are calculated by solving the stiff ODE system (\ref{eq:one})--(\ref{eq:twelve}). Since the right parts of the system are continuous functions with bounded derivatives, theoretically its solution will also be continuous and bounded. Therefore, discrepancy (\ref{eq:objective_func2}) will satisfy the Lipschitz condition with a priori unknown constant.

In the general form, the problem of the class specified above can be formulated mathematically as 
follows:
\begin{gather}
 \varphi^* = \varphi(y^\ast)=\min{\left\{\varphi(y):y\in D\right\}}, \label{problemN}\\
 D=\left\{y\in R^N: a_i\leq y_i \leq b_i, \;  1\leq i \leq N\right\} \label{D},
\end{gather}
where $a,b$ are the given vectors, $a,b\in R^N$, and the objective function $\varphi(y)$ satisfies the Lipschitz condition
\begin{equation}\label{Lip}
\left|\varphi(y_1)-\varphi(y_2)\right|\leq L\left\|y_1-y_2\right\|,\; y_1,y_2 \in D.
\end{equation}

The function $\varphi(y)$ is assumed to be multi-extremal and defined in the form of ``black box'' (i.e., in the form of some computing procedure, into the input of which the vector of parameters is supplied, and the corresponding function value is taken from the output). Moreover, each \textit{trial} (i.e., the computation of the function value at a point of the search domain) is assumed to be a time-consuming operation. 
As noted in Introduction, such a problem statement corresponds to the inverse problem of chemical kinetics completely.

Lipschitz condition (\ref{Lip}) can be utilized to estimate the global minimum of a function within an interval, and knowing the Lipschitz constant allows constructing global search algorithms and proving the convergence conditions for them (see, e.g., \cite{Strongin2000}).

The growth of computational costs with increasing the problem dimensionality is one of the main difficulties in solving multidimensional global optimization problems. Decreasing the number of trials at preserving the solution accuracy is possible by the complete utilization of some {\it a priori} assumptions on the objective function, which leads to adaptive serial optimization algorithms.

For example, the non-uniform space covering method \cite{Evtushenko2013} and the simplicial partitions method \cite{Zilinskas2010} are such methods. These approaches were successfully applied for the development of parallel optimization methods as well \cite{Evtushenko2009,Paulavicius2011}. 
Another adaptive approach to solving multidimensional problem (\ref{problemN}) is its reduction to a single one-dimensional problem or to several ones followed by the application of one-dimensional algorithms. 
Such a reduction can be made, for example, using the nested optimization scheme \cite{Grishagin2018} or Peano-Hilbert curves \cite{Barkalov2018}. 
The latter approach was used in the present work.

Using the continuous unambiguous mapping (Peano-Hilbert curve) $y(x)$ of the interval $[0,1]$ of the real axis on the hypercube $D$ from (\ref{D}), one can reduce multidimensional problem (\ref{problemN}) to a one-dimensional problem
\[
\varphi(y^\ast)=\varphi(y(x^\ast))=\min{\left\{\varphi(y(x)): x\in[0,1]\right\}},
\]
where the function $\varphi(y(x))$ will satisfy the uniform H{\"o}lder condition
\[
\left|\varphi(y(x_1))-\varphi(y(x_2))\right|\leq H\left|x_1-x_2\right|^{1/N}
\]
with the H{\"o}lder constant $H$ linked to the Lipschitz constant $L$ by the relation $ H=2 L \sqrt{N+3}$. 
The issues of the numerical construction of various approximations of the Peano-Hilbert curve were considered in \cite{Strongin2000,Sergeyev2013}.

So far, a search trial at some point $x'\in[0,1]$ will include first the construction of the image $y'=y(x')$ and then the computation of the value of the function $z'=\varphi(y')$.

%Данный комментарий можно и не включать...
%Note that the presence of rapidly decreasing components in the solution of the stiff ODE system (\ref{eq:one})--(\ref{eq:twelve}) can lead to the fact that in different subdomains the objective function can have different local Lipschitz constants. This problem will affect the convergence of any of the global optimization methods. To take it into account, a number of approaches have been proposed related to local tuning of global optimization methods (see, e.g., \cite{Barkalov2021,Strongin2020}).

\subsection{Parallel Asynchronous Global Search Algorithm}

In the approach proposed, the parallelization scheme corresponds to the ``master/worker'' principle.
In the master process, the global search algorithm is executed, it accumulates search information, evaluates the Lipschitz constant for the objective function on its base, determines new trial points and distributes them among worker processes. 

Worker processes receive the trial points from the master process, perform new trials at these points and send the trial results to the master process. 

Let us assume that the master process computes one point of the next trial at each iteration and sends it to the worker process for executing the trial. 
At the same time, the execution of the trial by the worker process is a much more computationally expensive operation than the choice of a new trial point by the master that excludes idle worker processes. 
In this case (unlike synchronous parallel algorithms), the total number of trials executed by each worker process will depend on the computational costs of executing a particular trial and cannot be estimated in advance.

In the description of the parallel algorithm, let us assume that $p+1$ computational processes are at our disposal: one master process and $p$ worker ones.
 
At the beginning of the search, the master process (let us assume it to be Process No 0) initiates the parallel execution of $p$ trials at $p$ different points of the search domain. 
Two of these points are boundary, while the rest are internal, i.e., at the points $\{y(x^1), y(x^2), ...,y(x^p)\}$ where 
$x^1 = 0$, $x^p = 1$, $x^i\in(0,1), i=2,..., p-1$.

Now let us assume that $k\geq 0$ trials (in particular, $k$ can be equal to 0) are completed, and worker processes perform trials at the points $\{y(x^{k+1}), y(x^{k+2}), ...,y(x^{k+p})\}$. 

Each worker process, having completed its trial at some point (without any loss of generality, let us assume this point to be $y(x^{k+1})$ corresponding to Process No 1), sends the trial result to the master process. 
In turn, the master process selects a new trial point $x^{k+p+1}$ for the worker process according to the rules described below.
Note that in this case we will have a set of preimages of the trial points
\[
I_k = \left\{ x^{k+1},x^{k+2},...,x^{k+p} \right\},
\]
at which the trials have already started, but have not yet been completed.

Step 1. Renumber the set of preimages of the trial points 
\[
X_k = \left\{x^1, x^2,...,x^{k+p} \right\},
\]
containing all preimages at which the trials either have been completed or underway in the increasing order (by the lower index) so that
\[
0=x_1<x_2<...<x_{k+p}=1.
\]
Step 2. Compute the values 
\begin{gather*}
M_1=\max \left\{ \frac{ \left|z_i - z_{i-1} \right|}{(x_i-x_{i-1})^{1/N}} : x_{i-1} \notin I_k, x_i \notin I_k, 2\leq i\leq k+p \right\},\\
M_2=\max \left\{ \frac{ \left|z_{i+1} - z_{i-1} \right|}{(x_{i+1}-x_{i-1})^{1/N}} : x_i \in I_k, 2\leq i < k+p \right\},\\
M=\max\{M_1,M_2\},
\end{gather*}
where $z_i=\varphi(y(x_i))$ if $x_i \notin I_k, \; 1\leq i \leq k+p$. The values $z_i$ at the points $x_i \in I_k$ are undefined since the trials at the points $x_i \in I_k$ have not yet been completed. If the value of $M$ equals 0, then set $M=1$.

Step 3. Juxtapose each interval $(x_{i-1},x_i), \; x_{i-1} \notin I_k, x_i \notin I_k, \; 2\leq i\leq k+p$ to the quantity $R(i)$, which is called the characteristic of the interval and is computed according to the formula
\begin{equation}\label{R}
R(i)=rM\Delta_i+\frac{(z_i-z_{i-1})^2}{rM\Delta_i}-2(z_i+z_{i-1}),
\end{equation}
where $\Delta_i=\left(x_i-x_{i-1}\right)^{1/N}$ and $r>1$ is the reliability parameter of the method.

Step 4. Select the interval $[x_{t-1},x_t]$, which the maximum characteristic corresponds to, i.e.,
\[
R(t) = \max \left\{ R(i): \; x_{i-1} \notin I_k, x_i \notin I_k, \; 2\leq i\leq k+p \right\}.
\]

Step 5. Define a new trial point $y^{k+p+1}=y(x^{k+p+1})$, the preimage of which is $x^{k+p+1} \in (x_{t-1},x_t)$ according to the formula
\[
x^{k+p+1} = \frac{x_{t}+x_{t-1}}{2} - \mathrm{sign}(z_{t}-z_{t-1})\frac{1}{2r}\left[\frac{\left|z_{t}-z_{t-1}\right|}{M}\right]^N.
\]

Upon computing the next trial point, the master process adds it to the set $I_k$ and sends it to the worker process, which initiates a new trial at this point. 

The master process terminates the algorithm if one of two conditions is satisfied: $\Delta_{t}<\epsilon$ or $k+p>K_{max}$.
The real number $\epsilon>0$ and the integer number $K_{max}>0$ are the parameters of the algorithm and correspond to the solution search precision and to the maximum number of trials, respectively.

The parallel asynchronous algorithm described above is based on the serial information global search algorithm. The theoretical substantiation of the algorithm convergence is given in \cite{Strongin2000}. The synchronous parallelization schemes used earlier in solving a number of applied problems \cite{Kalyulin2017,Modorskii2016} are also presented here.
The novelty of the present work lies in the practical implementation and application of the asynchronous parallelization scheme featured by a higher efficiency in solving problems with different computational costs for performing trials at different points of the search domain. 
It was confirmed by the results of the experiments described in the next section.

\section{Numerical Experiments}\label{Sec_Exp}

According to problem statements (\ref{eq:objective_func1}) and (\ref{eq:objective_func2}), the corresponding calculations were carried out in this work.
%ННГУ
The UNN supercomputer ``Lobachevsky'' (CentOS 7.2, SLURM, two CPUs Intel Sandy Bridge E5-2660 2.2 GHz and 64 Gb RAM on the node) was used for numerical experiments. The asynchronous global optimization algorithm was implemented using C++ (GCC 5.5.0 and Intel MPI were used); the objective function values were computed using Python 3.9.
The accuracy of solving the ODE system (\ref{eq:one})--(\ref{eq:twelve}) was set small enough so that the final error of the solution was much less than the accuracy of the stopping criterion of the optimization method and did not affect the method used.

\subsection{Search for activation energies and pre-exponential multipliers of the reaction}

First, the problem of searching for activation energies and pre-exponential multipliers of all reaction stages was solved. The number of optimized parameters is 30, i.e., two parameters for each of the fifteen reaction stages. For the activation energies, the search range was set to $0 \leq E_i \leq 100$ kJ/mol, and $0 \leq E_i \leq 10^{12}$ for the pre-exponential multipliers based on physicochemical considerations. However, based on the stiffness of the system of differential equations (\ref{eq:one}) and (\ref{eq:twelve}), no solution was found in such a wide range; this is due to the too high degree of pre-exponential multipliers included in the Arrhenius equation. Therefore, during the calculations, the value of the upper bound of the pre-exponential multipliers was reduced, and the solution was obtained at the upper bounds of $10^5$. The kinetic parameters found are presented in Table~\ref{table_res1}.

\begin{table}
\caption{Calculated rate constants of the reaction}
\label{table_res1}
\begin{center}
\begin{tabular}{cccccccc}
\hline
 & $k_1$ & $k_2$ & $k_3$ & $k_4$ & $k_5$ & $k_6$ & $k_7$\\
\hline\rule{0pt}{12pt}
$E$, kJ / mol & 98.60 & 98.22 & 99.07 & 2.29 & 97.61 & 94.30 & 8.67\\
$k^0$ & $1.46\cdot10^3$ & $1.53\cdot10^2$ & $1.03\cdot10^4$ & $1.10$ & $4.99\cdot10^2$  & $1.31\cdot10^2$ & $5.45\cdot10^2$\\
\hline
$k_8$ & $k_9$ & $k_{10}$ & $k_{11}$ & $k_{12}$ & $k_{13}$ & $k_{14}$ & $k_{15}$ \\
\hline\rule{0pt}{12pt}
11.57 & 73.16 & 65.54 & 13.86 & 4.93 & 2.69 & 21.19 & 0.63\\
$2.60\cdot10^2$ & $3.95\cdot10^4$ & $3.24\cdot10^4$ & $1.73\cdot10$ & $3.43$ & $1.78$ & $1.42\cdot10^3$ & $5.57\cdot10$\\[2pt]
\hline
\end{tabular}\end{center}\end{table}

%ННГУ
To evaluate the efficiency of the implemented optimization method, the results obtained  were compared applying the serial algorithm, the parallel synchronous and parallel asynchronous algorithms with the use of 8 nodes when solving the problem in the above statement.  
The minimum values of the objective function found by the respective methods, the time of solving the problem (in hours), and the values of speedup in time are presented in Table \ref{table_30D}. 
During the experiments, the parameter of the method $r=3.0$ from (\ref{R}) and the accuracy  $\epsilon = 10^{-3}\left\|b-a\right\|$ in the termination condition were used. Once the global search method achieved the preset accuracy, the solution was refined by the Hook-Jeeves local method \cite{HookJeeves} with the accuracy $\epsilon = 10^{-5}\left\|b-a\right\|$.

\begin{table}
\caption{Indicators achieved when solving the problem with 30 parameters}
\label{table_30D}
\begin{center}
\begin{tabular}{cccc}
\hline\noalign{\smallskip}
 Method      & Minimum  & Time (h.) & Speedup \\
\hline\noalign{\smallskip}
Serial       & 7.6   &    3.8     &  ---        \\
Synchronous  & 5.9   &   0.9     &   4.3       \\
Asynchronous & 4.8   &   1.1     &   3.6       \\
\noalign{\smallskip}\hline
\end{tabular}\end{center}\end{table}

The results show that both parallel algorithms demonstrated a moderate speedup, but found better solutions than the serial method. At that, the asynchronous algorithm found a better solution than the synchronous one. This also explains a smaller speedup of the asynchronous algorithm as compared with the synchronous one, since the asynchronous method executed more trials in the course of search for the optimal kinetic parameters of the problem.

%Часть УГНТУ
With the help of the obtained kinetic parameters, the direct problem of chemical kinetics was solved. However, the calculated kinetic curves poorly described the experimental data, which is confirmed by Fig. \ref{fig:res1}. It can be seen that the character of the calculated curve does not match the experimental dependence.

\begin{figure}
\begin{center}
  \includegraphics[width=0.7\linewidth]{res1.jpg}
  \caption{Concentration profiles of key components when calculating the activation energies and pre-exponential reaction multipliers according to problem statement~\ref{eq:objective_func1}. Temperature: 276.2 K. Symbols, experimental data; line, calculated values.}
  \label{fig:res1}  
\end{center}
\end{figure}

Therefore, it was decided to carry out calculations according to the second formulation of the problem (\ref{eq:objective_func2}), namely, to calculate the constants of each reaction stage separately, then, according to the Arrhenius dependence, find the kinetic parameters $E_i$ and $k_i^0$.

\subsection{Searching for rate constants separately for each temperature}

Next, the problem of finding the constants $k_i$ of the reaction stages included in the system (\ref{eq:one})--(\ref{eq:twelve}) was solved. Velocity constants were calculated for each of the temperatures (Table \ref{table_res2}).

\begin{table}[ht]
\caption{Calculated rate constants}\label{table_res2}
\begin{center}
\begin{tabular}{cccccccc}
\hline
T, K & $k_1$ & $k_2$ & $k_3$ & $k_4$ & $k_5$ & $k_6$ & $k_7$\\
\hline\rule{0pt}{12pt}
276.2 & 1.66 & 0.13 & 2.09 & 0.08 & 0.11 & 1.83 & 8.82 \\
279.2 & 1.94 & 0.33 & 2.19 & 0.32 & 0.16 & 1.96 & 14.15\\
282.2 & 2.14 & 0.99 & 5.43 & 2.77 & 0.56 & 2.13 & 45.85\\
286.2 & 2.43 & 0.99 & 5.51 & 2.81 & 0.56  & 2.35 & 58.97\\
\hline
$k_8$ & $k_9$ & $k_{10}$ & $k_{11}$ & $k_{12}$ & $k_{13}$ & $k_{14}$ & $k_{15}$ \\
\hline\rule{0pt}{12pt}
0.87 & 13.38 & 13.71 & 12.57 & 4.20 & 18.54 & 2.70 & 58.61\\
1.05 & 13.39 & 14.50 & 24.26 & 7.67 & 19.41 & 3.86 & 74.64\\
1.09 & 18.92 & 17.41 & 33.73 & 6.07 & 19.24 & 9.61 & 77.37\\
1.50 & 24.09 & 17.88 & 34.22 & 4.35 & 23.04 & 16.38 & 94.45\\[2pt]
\hline
\end{tabular}\end{center}\end{table}

%ННГУ
When solving this problem, the operation performances of the serial algorithm, of the parallel synchronous and parallel asynchronous algorithms were also compared.  
The minimum values of the objective function found by the respective methods, the time of solving the problem (in hours), and the speedup in time with the use of 8 nodes are presented in Table \ref{table_15D}. All the parameters of the method were the same as in the previous run. 

\begin{table}
\caption{Indicators achieved when solving the problem with 15 parameters}
\label{table_15D}
\begin{center}
\begin{tabular}{cccc}
\hline\noalign{\smallskip}
 Method      & Minimum  & Time (h.) & Speedup \\
\hline\noalign{\smallskip}
Serial       & 0.35   &   3.4     &  ---        \\
Synchronous  & 0.36   &   0.4     &   8.6       \\
Asynchronous & 0.35   &   0.2     &   16.6       \\
\noalign{\smallskip}\hline
\end{tabular}\end{center}\end{table}

The results show that all algorithms found good solutions (in the values of the objective function). At that, the asynchronous algorithm demonstrated twice as much speedup than the synchronous one. The good speedup of the asynchronous algorithm remains with a larger number of nodes; the results of its work are presented in Table \ref{tab_parall}.

Note that when solving the global optimization problem, the number of iterations of the parallel algorithm (and hence its speedup) significantly depends on the estimation of the Lipschitz constant of the objective function. The constant is adaptively estimated during the work of the algorithm and may vary depending on the accumulated search information. With a correct estimate of the Lipschitz constant, the method can converge to the global optimum point faster than in the case of its incorrect estimate. This explains the effect of superlinear speedup observed in the experiments.

\begin{table}
\caption{Speedup of the asynchronous parallel algorithm}
\label{tab_parall}
\begin{center}
\begin{tabular}{cccc}
\hline\noalign{\smallskip}
Nodes  & Minimum  & Time (h.) & Speedup \\
\hline\noalign{\smallskip}
1  & 0.35   &   3.4     &   ---        \\
8  & 0.35   &   0.2     &   16.6       \\
16 & 0.36   &   0.1     &   34.3       \\
32 & 0.35   &   0.06    &   59.5       \\
\noalign{\smallskip}\hline
\end{tabular}\end{center}\end{table}

%Часть УГНТУ
The direct problem of chemical kinetics was solved with the found rate constants, the results of comparison with the experimental data are shown in Fig. \ref{fig:res2} (an experiment at a temperature of 276.2 K is presented). We see that this time the description of the data turned out to be rather accurate.

\begin{figure}
\begin{center}
  \includegraphics[width=0.7\linewidth]{res2.jpg}
  \caption{Concentration profiles of key components when calculating the rate constants according to problem statement~\ref{eq:objective_func2}. Temperature: 276.2 K. Symbols, experimental data; line, calculated values.}
  \label{fig:res2}  
\end{center}
\end{figure}

After calculating the rate constants for different temperature values, it is possible to calculate the parameters $E_i$ and $k_i^0$. Thus, for example, for the first stage of the reaction $E_1 = 24.65$ kJ/mol, $k_1^0 = 6.85 \cdot 10^4$ min$^{-1}$, for the third stage $E_3 = 75.57$ kJ/mol, $k_3^0 = 2.62 \cdot 10^{14}$ kg $\cdot$mol$^{-1}\cdot$min$^{-1}$. However, for some stages, the parameter values $k^0$ were fairly high: for the second stage $k_2^0 = 8.52 \cdot 10^{25}$ min$^{-1}$, for the seventh stage $k_7^0 = 2.82 \cdot 10^{26}$ kg mol$^{-2}$ min$^{-2}$. Therefore, in future works, the constants corresponding to these kinetic parameters will be recalculated.
Since the problem is multidimensional, there may be several solutions that describe experimental data within a certain deviation error. The solution may change if, for example, additional experimental data are found. However, the algorithm proposed in the paper provides a single solution within the search range, and the high values of the pre-exponential multipliers explain the presence of complex stages, which, in turn, consist of several stages that can be refined by chemists. The traditional scheme of assessing the quality of the problem solution as a deviation of the found solution from the exact one is not applicable here since the exact solution is not known.

\section{Conclusions and Future Work}
The article describes the search for the kinetic parameters of the industrial chemical reaction of isobutane alkylation with olefins in the presence of sulfuric acid. The search was carried out for the rate constants of all reaction stages, and the activation energies and pre-exponential multipliers were calculated. The optimization method allowed us to find a fairly accurate description of the experimental data. In the future, it is planned to use this method and parallelization to eliminate the high values of the pre-exponential reaction multipliers and search for optimal conditions for the alkylation reaction using the developed kinetic model.

\medskip

\textbf{Acknowledgments}. This study was supported by the Russian Science Foundation, project No.\,21-11-00204, and the Russian Foundation for Basic Research, project No.\,19-37-60014.

%
% ---- Bibliography ----
%
\bibliographystyle{spmpsci}
\bibliography{bibliography}{}

\end{document}
