\documentclass[11pt, oneside, a4paper]{article}
\usepackage[utf8]{inputenc}
%\usepackage[cp1251]{inputenc} % кодировка
\usepackage[english, russian]{babel} % Русские и английские переносы
\usepackage{graphicx}          % для включения графических изображений
\usepackage{cite}              % для корректного оформления литературы
\usepackage{pavt-ru}                                

\begin{document}

% \title - название статьи
% \authors - список авторов

\title{Параллельный алгоритм минимизации многоэкстремальных функций, имеющих разрывы \footnote{Исследование выполнено за счет гранта Российского научного фонда (проект № 16-11-10150).}}

\authors{К.А.~Баркалов, М.А.~Усова}
\organizations{Нижегородский государственный университет им. Н.И. Лобачевского}

В рамках начатого исследования рассматривается параллельный алгоритм глобальной оптимизации для функций, имеющих одну или несколько точек разрыва типа конечного скачка. Разрывы целевой функции могут являться отражением в математической модели специфики оптимизируемого объекта (например, ударными воздействиями, резонансными явлениями, скачками геометричских размеров или свойств материала и т.п.). Во многих случаях множество точек разрыва является известным. Однако вместе с тем существуют задачи, в которых нет априорных оценок точек разрыва, но известно, что такие точки возможны. 

Задача рассматривается в одномерной постановке
\[
\varphi^* = \varphi(x^*)=\min\left\{\varphi(x):x\in[a,b]\right\},
\] 
т.к. решение многомерных задач может быть сведено к решению соответствующих им одномерных задач \cite{Grishagin07}.

Метод распараллен с использованием общего подхода, изложенного в \cite{Strongin13} и успешно примененного для распараллеливания оптимизационных алгоритмов для различных архитектур \cite{Barkalov14,Barkalov16}.

Проведены предварительные эксперименты с распараллеливанием на общей памяти.
Результаты экспериментов показали ...
Направлением дальнейших работ будет (i) расширение класса рассматриваемых задач (многомерные задачи с невыпуклыми ограничениями); (ii) распараллеливание алгоритма с использованием распределенной памяти.


\begin{biblio}

\bibitem{Strongin91}
Стронгин~Р.Г. Поиск глобального оптимума. М.: Знание, 1990. 48~с.

\bibitem{Grishagin07}
Городецкий~С.Ю., Гришагин~В.А. Нелинейное программирование и многоэкстремальная оптимизация. Н.~Новгород: Изд-во ННГУ, 2007. 489~с.

\bibitem{Strongin13}
Стронгин~Р.Г., Гергель~В.П., Гришагин~В.A., Баркалов~К.А. Параллельные вычисления в задачах глобальной оптимизации. М.: Издательство Московского университета, 2013. 280~с.

\bibitem{Barkalov14}
Баркалов~К.А. Использование параллельных характеристических алгоритмов для решения многомерных задач глобальной оптимизации  // Вестник ЮУрГУ. Серия: Вычислительная математика и информатика. 2014. Т.~3, №~4. С.~116--123.

\bibitem{Barkalov16}
Баркалов~К.А., Лебедев~И.Г., Соврасов~В.В., Сысоев~А.В. Реализация параллельного алгоритма поиска глобального экстремума функции на Intel Xeon Phi // Вычислительные методы и программирование. 2016. Т.~17, №~1. С.~101--110.

%\bibitem{sokolinsky}
%Соколинский~Л.Б. Организация параллельного выполнения запросов в многопроцессорной машине баз данных с иерархической архитектурой // Программирование. 2001. №~6. С.~13--29.

%\bibitem{stonebraker}
%Stonebraker~M., Kemnitz~G. The POSTGRES Next-generation Database Management System // Communications of the ACM. 1991. Vol.~34, No.~10. P.~78--92. DOI:~\href{http://dx.doi.org/10.1145/125223.125262}{10.1145/125223.125262}.

\end{biblio}
\end{document}