% This is samplepaper.tex, a sample chapter demonstrating the
% LLNCS macro package for Springer Computer Science proceedings;
% Version 2.20 of 2017/10/04
%
\documentclass[runningheads]{llncs}
%
\usepackage{graphicx}
\usepackage{amssymb}
\usepackage{amsmath}
\usepackage{subfig}
\usepackage{cite}

\DeclareMathOperator*{\argmax}{arg\,max}
\DeclareMathOperator*{\argmin}{arg\,min}
\DeclareMathOperator{\sign}{sign}

% Used for displaying a sample figure. If possible, figure files should
% be included in EPS format.
%
% If you use the hyperref package, please uncomment the following line
% to display URLs in blue roman font according to Springer's eBook style:
% \renewcommand\UrlFont{\color{blue}\rmfamily}

\begin{document}
%
\title{Comparison of Several Stochastic and Deterministic Derivative-free Global Optimization Algorithms\thanks{The study was supported by the Russian Science Foundation, project No 16-11-10150.}}
%
%\titlerunning{Abbreviated paper title}
% If the paper title is too long for the running head, you can set
% an abbreviated paper title here
%
\author{Vladislav Sovrasov}
%
\authorrunning{V. Sovrasov}
%
\institute{Lobachevsky State University of Nizhni Novgorod, Russia \\
  \email{sovrasov.vlad@gmail.com}
}%
\maketitle              % typeset the header of the contribution
%

\begin{abstract}
In this paper popular open-source solvers are compared against
Globalizer solver, which is developed at the Lobachevsky State University.
The Globalizer is designed to solve problems with black-box objectives satisfying the Lipschitz condition and shows
competitive performance with other similar solvers. The comparison is done on several sets of
challenging multi-extremal benchmark functions. Also this work considers a method of
heuristic hyperparameters control for the Globalizer allowing to reduce amount of
initial tuning before optimization. The proposed scheme allows substantially increase convergence speed
of the Globalizer by switching between ``local'' and ``global'' search phases in runtime.
% We would like to encourage you to list your keywords within
% the abstract section using the \keywords{...} command.
\keywords{deterministic global optimization $\cdot$ stochastic global optimization
  $\cdot$ algorithms comparison $\cdot$ derivative-free algorithms $\cdot$ black-box
optimization $\cdot$ multi-extremal problems}
\end{abstract}
%
\section{Introduction}

The problem of finding the global minima of the nonlinear nonconvex functions is considered to be
one of the most difficult mathematical programming problems traditionally. Often, it appears to be
more complex than the local optimization in an essentially multidimensional space.
For the latter, the application of the simplest gradient descent method or of the pattern search
algorithms may appear to be sufficient \cite{torczon1997}
whereas in order to \textit{guarantee} the finding of the global optimum, the optimization methods
have to
accumulate the information on the behavior of the objective function in the whole search domain
\cite{Jones2009,Paulavicius2011,Evtushenko2013,strSergGO}.
Recently, various stochastic global optimization algorithms, first of all, the evolution ones
\cite{Storn1997, SCHLUTER2009, KennedyEberhart1995} became popular. These ones have rather
simple structure and allow solving the problems of large dimensionality. However, these methods
provide the global convergence in the probabilistic meaning only.

In the present work, the open-source implementations of the eight different global optimization
methods included into the NLOpt library \cite{nlopt} and the SciPY package \cite{scipy} are
considered.
All algorithms were tested on a set of 900 essentially multiextremal functions, which has been
generated with the use of special problem generators \cite{Gaviano2003, grishaginClass}.

\section{Related Work}

Earlier, the comparison of the stochastic global optimization algorithms \cite{Ali2005, JSSv060i06}
as well as of the deterministic ones \cite{posik2012, KVASOV2018245, Liberti2005} between each
other has been considered in the literature. In these works, most of modern methods have
been studied in details. In the majority of works, the sets of well-known test problems (for example, the Rastrigin
function, Ackley function, etc.) were taken as the sets of test functions. The sizes of such sets don't
exceed 100 different functions usually, some of which can be the single-extremal ones (such as the
Rosenbrock function).

In Ref. \cite{Beiranvand2017}, some general principles were formulated, which, in the author's
opinion, should be obeyed when comparing the optimization methods. In particular, the authors say
about the advantages of the problem generators allowing generating the large sets of problems thus
minimizing the random effects when comparing the methods. At the same time, the use of a single
generator can appear to be not enough for a comprehensive comparison of the methods. In order to
overcome this problem in part, the authors of Ref. \cite{Beiranvand2017} advise to use several
generators of various nature and to create the sets of problems of various complexity.

Taking into account the experience of the preceding works in the field of comparison of the
optimization methods, two generators of the test problems of different nature will be used in the
present work. Using these ones, 9 sets of 100 problems of various complexity with the
dimensionality varying from 2 to 5 were generated.

\section{Statement of Multidimensional Global Optimization Problem}
In this paper, the core class of optimization problems, which can be solved using
global optimization methods, is formulated. This class involves the multidimensional global
optimization problems without constraints, which can be defined in the following way:
\begin{equation}
\label{eq:task}
\begin{array}{cr}\\
  \varphi(y^*)=\min\{\varphi(y):y\in D\}, \\
  D=\{y\in \mathbb{R}^N:a_i\leq y_i\leq{b_i}, 1\leq{i}\leq{N}\}
\end{array}
\end{equation}
with the given boundary vectors  $a$ and  $b$. It is supposed, that the objective function
\(\varphi(y)\) satisfies the Lipschitz condition
\begin{equation}
\label{eq:lip}
|\varphi(y_1)-\varphi(y_2)|\leq L\Vert y_1-y_2\Vert,y_1,y_2\in D,
\end{equation}
where \(L>0\) is the Lipschitz constant, and \(||\cdot||\) denotes the norm in \(\mathbb{R}^N\)
space.
\par
Usually, the objective function \(\varphi(y)\) is defined as a computational procedure,
according to which the value \(\varphi(y)\) can be calculated for any vector \(y\in D\)
(let us further call such a calculation \textit{a trial}). It is supposed that this procedure
is time-consuming.

\section{Review of Considered Optimization Methods}

\subsection{Algorithm of Global Search}
\label{sub:ags}
\subsubsection{Dimension Reduction with Space-Filling Curves}
Within the framework of the information-statistical global optimization theory,
the Peano space-filling curves (or \textit{evolvents}) \(y(x)\) mapping the interval \([0,1]\)
onto an \(N\)-dimensional hypercube \(D\) unambiguously are used for the dimensionality
reduction \cite{sergeyevStronginLera2013, strongin1978, strSergGO}.
\par
As a result of the reduction, the initial multidimensional global optimization
problem (\ref{eq:task}) is reduced to the following one-dimensional problem:
\begin{equation}
\label{eq:oneDimTask}
\varphi(y(x^*))=\min\{\varphi(y(x)):x\in [0,1]\}.
\end{equation}
\par
It is important to note that this dimensionality reduction scheme transforms the % minimized
Lipschitzian function from (\ref{eq:task}) to the corresponding one-dimensional
function \(\varphi(y(x))\), which satisfies the uniform H{\"o}lder condition, i. e.
\begin{equation}
\label{eq:holder}
|\varphi(y(x_1))-\varphi(y(x_2))|\leq H{|x_1-x_2|}^{\frac{1}{N}}, x_1,x_2\in[0,1],
\end{equation}
where the constant $H$ is defined by the relation \(H=2L\sqrt{N+3}\), \(L\) is the Lipschitz
constant from (\ref{eq:lip}), and \(N\) is the dimensionality of the optimization problem
(\ref{eq:task}).
\par
The algorithms for the numerical construction of the Peano curve approximations are
given in \cite{strSergGO}.

\par
The computational scheme obtained as a result of the dimensionality reduction consists of the
following:
\begin{itemize}
  \item The optimization algorithm performs the minimization of the reduced one-dimensional
  function \(\varphi(y(x))\) from (\ref{eq:oneDimTask}),
  \item After determining the next trial point \(x\), a multidimensional image \(y\) is calculated by
using the mapping \(y(x)\),
  \item The value of the initial multidimensional function \(\varphi(y)\) is calculated at the point
\(y\in D\),
  \item The calculated value \(z=\varphi(y)\) is used further as the value of the reduced one-
dimensional function \(\varphi(y(x))\) at the point \(x\).
\end{itemize}

Optimization method applied in Globalizer \cite{globalizerSystem} to solve the reduced problem
(\ref{eq:oneDimTask}) is based on the AGS method, which can be presented as follows ---
see \cite{strongin1978}, \cite{strSergGO}.
\par
The algorithm considered for solving the stated problem implies generating
a sequence of points \(x_k\), in which the values of the minimized function \(z_k = f(x_k)\)
are computed. Let us call the process of computating the function value
(including calculating an image \(y^k=y(x^k)\)) a trial, and the pair \((x^k,z^k)\) ---
the result of the trial. A set of the pairs \(\{(x^k,z^k)\}, 1\leqslant k\leqslant n\)
makes up the search information accumulated by the method after executing \(n\) steps.
\par
The initial iteration of the algorithm is performed at an arbitrary point \mbox{\(x^1\in(0,1)\)}.
Then, let us suppose that \(k\), \(k\ge 1\), optimization iterations have been completed already.
The selection of the trial point \(x^{k+1}\) for the next iteration is performed according to the
following rules.

\textit{Step 1.} Renumber the points in the set \(X_k=\{x^1,\dotsc,x^k\}\cup\{0\}\cup\{1\}\),
which includes the boundary points of the interval \([0,1]\) as well as the points of
preceding trials, by the lower indices in order of increasing coordinate values  i.e.
\begin{displaymath}
0=x_0<x_1<\dotsc<x_{k+1}=1
\end{displaymath}
\par
\textit{Step 2.} Assuming \(z_i=f(x_i),1\leqslant i\leqslant k\), compute the values
\begin{equation}
\label{eq:step2}
\mu=\max_{1\leqslant i\leqslant k}\dfrac{|z_i-z_{i-1}|}{\Delta_i},
\begin{matrix}
    M =
    \left\{
    \begin{matrix}
    r\mu,\mu>0 \\
    1,\mu=0
    \end{matrix} \right.
    \end{matrix}
\end{equation}
where \(r>1\) is a predefined parameter for the method, and \(\Delta_i=(x_i-x_{i-1})^\frac{1}{N}\).
\par
\textit{Step 3.} For each interval \((x_{i-1},x_i),1\leqslant i\leqslant k+1\), compute the
characteristics according to the formulae
\begin{equation}
\label{step3_1}
R(1)=2\Delta_1-4\dfrac{z_1}{M},R(k+1)=2\Delta_{k+1}-4\dfrac{z_k}{M},
\end{equation}
\begin{equation}
\label{eq:step3_2}
R(i)=\Delta_i+\dfrac{(z_i-z_{i-1})^2}{M^2\Delta_i}-2\dfrac{z_i+z_{i-1}}{M},1<i<k+1.
\end{equation}
\par
\textit{Step 4.} Determine the interval with the maximum characteristic \((x_{t-1}, x_t),\:t=\argmax_{1\leqslant i \leqslant k+1}R(i)\)

\textit{Step 5.} Execute a new trial at point \(x_{k+1}\) computed according to the formula
\begin{displaymath}
x_{k+1}=\dfrac{x_{t}+x_{t-1}}{2},t=1,t=k+1,
\end{displaymath}
\begin{equation}
\label{step5}
x_{k+1}=\dfrac{x_{t}+x_{t-1}}{2}-\sign(z_{t}-z_{t-1})\dfrac{1}{2r}\left[\dfrac{|z_{t}-z_{t-1}|}{\mu}\right]^N,1<t<k+1.
\end{equation}

The stopping condition, which terminated the trials, is defined by the inequality
\(\Delta_{t}\leqslant \varepsilon\)
for the interval with the maximum characteristics from Step 4 and \(\varepsilon >0\) is the
predefined accuracy of the optimization problem solution. If the stopping condition is not satisfied,
the index \(k\) is incremented by \(1\), and the new global optimization iteration is executed.

The convergence conditions of the described algorithm are given, for example, in \cite{strSergGO}.

\subsubsection{Hyperparameters control in AGS}

The parameter $r$ from (\ref{eq:step2}) affects the global convergence of AGS directly (see
\cite{strSergGO}, Chapter 8):
at high enough value of $r$, the method converges to all global minima of the objective function with
guarantee.
At the same time, according to (\ref{eq:step3_2}) and (\ref{step5}), at the infinitely high value of $r$, AGS turns into
the brute force search method on a uniform grid.

In the ideal case, in order to provide the highest convergence speed, the estimate of the Lipschitz
constant from (\ref{eq:step2})
should not be too overestimated, but in practice the actual value of $L$ from (\ref{eq:lip}) in
unknown, and one has either to take an obviously overestimated value of $r$ or to execute several
runs of AGS with different parameters. In order to resolve the problem of choosing $r$ to some extent,
let us use the following scheme:
\begin{itemize}
  \item execute $q$ iterations of AGS with $r=r_{max}$;
  \item execute $q$ iterations of AGS with $r=r_{min}$;
  \item repeat the above steps either until convergence or until the allowed number of iterations are
exhausted.
\end{itemize}

In the above algorithm, $r_{min} < r_{max}$, $q > 1$. Instead one parameter $r$, now
3 ones should be selected. However, according to the results of the numerical experiments, it is easier
than to find the optimal value of $r$.
Intuitively, the practical efficiency of the proposed scheme can be explained by the fact that now the
operation of the method takes place in two modes: the global search with $r=r_{max}$ and the local
one with $r=r_{min}$. If during the global search phase, the method approached the global minimum
whereas during the next phase, the estimate of the global minimum  would be refined rapidly.
If two phases are not enough, the process is continued. This way, a better trade-off
between the exploration and the exploitation is achieved.
Further, we will denote the method utilizing the scheme described above as AGS-AR.

\subsection{Other Optimization Methods}
\begin{itemize}
  \item \textbf{Multi Level Single Linkage} \cite{Kan1987StochasticGO}. MLSL is an improved
multistart algorithm.
  It samples low-discrepancy starting points and does local optimizations from them. In contrast to
the dummy multistart schemes
  MLSL uses some clustering heuristics to avoid multiple local descents to already explored local
minima.

  \item \textbf{DIRECT} \cite{Jones2009}. The algorithm is deterministic and recursively divides
the search space and forms a tree of hyper-rectangles (boxes). DIRECT uses the objective function
values and the Lipschitz condition (\ref{eq:lip}) to estimate promising boxes.

  \item \textbf{Locally-biased DIRECT (DIRECT$l$)} \cite{Gablonsky2001}. It's a variation of
DIRECT which pays less attention to non-promising boxes and therefore
  has less exploration power: it can converge faster on problems with few local minima, but lost the
global one in complicated cases.

  \item \textbf{Dual Simulated Annealing} \cite{XIANG1997216}. This stochastic method is a
combination of the Classical Simulated Annealing and the Fast Simulated Annealing coupled to a
strategy for applying a local search on accepted locations. It converges much faster than both parent
algorithms, CSA and FSA.

  \item \textbf{Differential Evolution} \cite{Storn1997}. DE is an adaptation of the original genetic
algorithm to
  the continuous search domain.

  \item \textbf{Controlled Random Search} \cite{Price1983}. The CRS starts with a set of random
points and then defines
  the next trial point in relation to a simplex chosen randomly from a stored configuration of points.
CRS in not an
  evolutional algorithm, although stores something like population and performs transformation
resembling a mutation.

  \item \textbf{StoGO} \cite{Madsen1998}. StoGO is dividing the search space into smaller hyper-
rectangles via a branch-and-bound approach,
  and searching them by a local-search algorithm, optionally including some randomness.

\end{itemize}

All the mentioned algorithms are available in source codes as parts of wide-spread optimization packages.
DIRECT, DIRECT$l$, CRS, MLSL and StoGO are part of the NLOpt library \cite{nlopt}.
Differential Evolution and DSA can be found in
the latest version of the SciPy \cite{scipy} package for Python.

\section{Tools for Comparison of Global Optimization Algorithms}

The use of the sets of test problems with known solutions generated by some random mechanisms is
one of commonly accepted approaches to comparing the optimization algorithms
\cite{Beiranvand2017}. In the present work, we will use two generators of test problems generating
the problems of different nature \cite{grishaginClass, Gaviano2003} \footnote{Software implementations of
these generators are available in source codes at the page \url{https://github.com/sovrasov/global-optimization-test-problems}}.

Let us denote the problem set obtained with the use of the first generator from \cite{grishaginClass}
as \(F_{GR}\). The mechanism of generation of the problems \(F_{GR}\) doesn't provide the
control of the problem complexity and of the number of local optima. However, the generated
functions are known to be the multiextremal ones essentially. Besides, the problems generated by
\(F_{GR}\) are the two-dimensional ones. In the present work, we will use 100 functions from the
class \(F_{GR}\) generated randomly.

The GKLS generator \cite{Gaviano2003} allows obtaining the problems of given dimensionality
with given number of extrema. Moreover, GKLS allows adjusting the complexity of the problems by
decreasing or increasing the size of the global minimum attractor. In
\cite{SergeyevKvasov2006} the parameters of the generator allowing generating the sets of 100
problems each of two levels of complexity (Simple and Hard) of the dimensionality equal to 2, 3, 4,
and 5 are given. Following the authors of the GKLS generator, we will use the parameters proposed
by them and, this way, add 800 more problems of various dimensionalities and complexity into the
test problem set.

Let us suppose a test problem to be solved if the optimization method executes the scheduled trial
\(y^k\) in a \(\delta\)-vicinity of the global minimum \(y^*\), i.e. $\left\|y^k-y^*\right\|\leq \delta
= \alpha\left\|b-a\right\|$, where \(a\) and \(b\) are the left and the right boundaries of the hypercube
from (\ref{eq:task}), $\alpha$ is relative precision. If this relation is not fulfilled before the expiration of the limit of the number of
trials, the problem was considered to be unsolved. The limit of the number of trials and $\alpha$ were set
for each problem class according to the dimensionality and complexity (see Table \ref{tab:limits}).

\begin{table}
\begin{center}
\caption{Trials limits and relative precision for the test problem classes}
  \begin{tabular}{|l|{c}|{c}|}
    \hline
  Problems class & Trials limit & $\alpha$\\
  \hline
  \(F_{GR}\) & 5000 & 0.01 \\
  \hline
  GKLS 2d Simple & 8000 & 0.01 \\
  \hline
  GKLS 2d Hard & 9000 & 0.01 \\
  \hline
  GKLS 3d Simple & 15000 & 0.01 \\
  \hline
  GKLS 3d Hard & 25000 & 0.01 \\
  \hline
  GKLS 4d Simple & 150000 & $\sqrt[4]{10^{-6}}$ \\
  \hline
  GKLS 4d Hard & 250000 & $\sqrt[4]{10^{-6}}$ \\
  \hline
  GKLS 5d Simple & 350000 & $\sqrt[5]{10^{-7}}$ \\
  \hline
  GKLS 5d Hard & 600000 & $\sqrt[5]{10^{-7}}$ \\
  \hline
  \end{tabular}
  \label{tab:limits}
\end{center}
\end{table}

Let us consider the averaged number of trials executed to solve a single problem and the number of
solved problems as the characteristics of the optimization method on each class. The less the number
of trials, the faster the method converges to a solution, hence the less times it turns to a potentially
computation-costly procedure of computing the objective function. The number of solved problems
evidences the reliability of the method at given parameters on the class of test problems being
solved. In order to make independent the quantities featuring the reliability and the speed of convergence,
averaged number of trials always was calculated taking into account solved problems only.

The average number of trials doesn't represent the real behavior of an optimization method
on a problems set in some cases. For an instance, if a method performs well on the most problems
and spends too much trials to solve the least several problems, we wouldn't catch such
case looking at the average number of trials only.
As an advanced measure of performance we will use the operating characteristic \cite{grishaginClass}.
It's defined by a set of points on the \((K, P)\) plane where \(K\) is the average number of search trials
conducted before satisfying the termination condition when minimizing a function
from a given class, and \(P\) is the proportion of problems solved successfully.
If at a given \(K\), the operating characteristic of a method goes higher than one
from another method, it means that at fixed search costs, the former method has a
greater probability of finding the solution. If some value of \(P\) is fixed, and the
characteristic of a method goes to the left from that of another method, the former
method requires fewer resources to achieve the same reliability.

\section{Results of Numerical Experiments}
\label{sec:experiments}
The results of various algorithms on different problem classes depend on the adjustments of
algorithms directly. In most cases, the authors of software implementations are oriented onto the
problems of medium difficulty. In order to obtain a satisfactory result when solving the essentially
multiextremal problems, a correction of some parameters is required. When conducting the
comparison, the following parameters for the methods were employed:
\begin{itemize}
  \item in the AGS-AR method, the parameter of alternation the
  global and local stages $q$ was set to be equal to $50\cdot\log_2(N-1)\cdot N^2$, also $r_{min}=3,\:r_{max}=2\cdot r_{min}$;
  \item in the DIRECT and DIRECT\(l\) methods, the parameter \(\epsilon=10^{-4}\);
  \item in the SDA method, the parameter \(visit=2.72\).
\end{itemize}

The rest parameters were varied subject to the problem class (see Table \ref{tab:params}).
For the AGS the value ot the $r$ parameter, such that the method solves all problems and performs the minimum amount of trials,
was estimated by brute force on the uniform grid with step $0.1$.

\begin{table}
\begin{center}
\caption{Class-specific parameters of the optimization algorithms}
  \begin{tabular}{|l|{c}|{c}|{c}|}
    \hline
    & AGS & CRS & DE\\
  \hline
  \(F_{GR}\) & \(r=3\) & popsize=150 & mutation=(1.1,1.9), popsize=60 \\
  \hline
  GKLS 2d Simple & \(r=4.6\) & popsize=200 & mutation=(1.1,1.9), popsize=60 \\
  \hline
  GKLS 2d Hard & \(r=6.5\) & popsize=400 & mutation=(1.1,1.9), popsize=60 \\
  \hline
  GKLS 3d Simple & \(r=3.7\) & popsize=1000 & mutation=(1.1,1.9), popsize=70 \\
  \hline
  GKLS 3d Hard & \(r=4.4\) & popsize=2000 & mutation=(1.1,1.9), popsize=80 \\
  \hline
  GKLS 4d Simple & \(r=4.7\) & popsize=8000 & mutation=(1.1,1.9), popsize=90 \\
  \hline
  GKLS 4d Hard & \(r=4.9\) & popsize=16000 & mutation=(1.1,1.9), popsize=100 \\
  \hline
  GKLS 5d Simple & \(r=4\) & popsize=25000 & mutation=(1.1,1.9), popsize=120 \\
  \hline
  GKLS 5d Hard & \(r=4\) & popsize=30000 & mutation=(1.1,1.9), popsize=140 \\
  \hline
\end{tabular}
  \label{tab:params}
\end{center}
\end{table}

\begin{table}
\begin{center}
\caption{Averaged number of trials executed by optimization methods for solving the test
optimization problems}
\resizebox{\textwidth}{!}{%
  \begin{tabular}{|l|{c}|{c}|{c}|{c}|{c}|{c}|{c}|{c}|{c}|{c}|}
    \hline
    & AGS & AGS-AR & CRS & DIRECT & DIRECT\(l\) & MLSL & SDA & DE & StoGO \\
  \hline
  \(F_{GR}\)     & 193.1 & 248.3 & 400.3 & \textbf{182.2} & 214.9 & 947.2 & 691.2 & 1257.3 & 1336.8 \\
  \hline
  GKLS 2d Simple & 254.9 & 221.6 & 510.6 & \textbf{189.0} & 255.2 & 556.8 & 356.3 & 952.2 & 1251.5 \\
  \hline
  GKLS 2d Hard   & \textbf{728.7} & 785.0 & 844.7 & 985.4 & 1126.7 & 1042.5 & 1637.9 & 1041.1 & 2532.2 \\
  \hline
  GKLS 3d Simple &  1372.1 & 1169.5 & 4145.8 & \textbf{973.6} & 1477.8 & 4609.2 & 2706.5 & 5956.9 & 3856.1 \\
  \hline
  GKLS 3d Hard   &  3636.1 & \textbf{1952.1} & 6787.0 & 2298.7 & 3553.3 & 5640.1 & 4708.4 & 6914.3 & 7843.2 \\
  \hline
  GKLS 4d Simple &  5729.8 & \textbf{4919.1} & 19883.6 & 7328.8 & 15010.0 & 41484.8 & 22066.0 & 6271.2 & 29359.2 \\
  \hline
  GKLS 4d Hard   &  13113.4 & \textbf{12860.1} & 27137.4 & 22884.4 & 55596.1 & 80220.1 & 68048.0 & 12487.6 & 58925.5  \\
  \hline
  GKLS 5d Simple &  \textbf{5821.5} & 6241.3 & 62921.7 & 5966.1 & 10795.5 & 52609.2 & 34208.8 & 20859.4 & 69206.8 \\
  \hline
  GKLS 5d Hard   &  \textbf{17008.6} & 21555.1 & 87563.9 & 61657.3 & 148637.8 & 138011.8 & 115634.6 & 26850.0 & 141886.5 \\
  \hline
\end{tabular}}
  \label{tab:trials}
\end{center}
\end{table}

The results of running the optimization methods on the considered problem classes are presented in
Tables \ref{tab:trials}, \ref{tab:solved}. The DIRECT, AGS and AGS-AR methods have demonstrated the
best convergence speed on all classes, at that AGS-AR inferior to DIRECT on the 2d problems from the
Simple classes and has an advantage on the problems of the Hard classes. As one can see from Table
\ref{tab:solved}, the deterministic methods (AGS, AGS-AR, DIRECT, and DIRECT\(l\)) were the
most reliable. Among the stochastic methods, MLSL and SDA have demonstrated the highest
reliability.

\begin{table}
\begin{center}
\caption{Number of test optimization problems solved by the methods}
  \begin{tabular}{|l|{c}|{c}|{c}|{c}|{c}|{c}|{c}|{c}|{c}|{c}|}
    \hline
    & AGS & AGS-AR & CRS & DIRECT & DIRECT\(l\) & MLSL & SDA & DE & StoGO \\
  \hline
  \(F_{GR}\)     &  100 & 100 & 76 & 100 & 100 & 97 & 96 & 96 & 67\\
  \hline
  GKLS 2d Simple &  100 & 100 & 85 & 100 & 100 & 100 & 100 & 98 & 90\\
  \hline
  GKLS 2d Hard   &  100 & 97 & 74 & 100 & 100 & 100 & 93 & 85 & 77 \\
  \hline
  GKLS 3d Simple &  100 & 100 & 75 & 100 & 100 & 100 & 89 & 86 & 44 \\
  \hline
  GKLS 3d Hard   &  100 & 100 & 72 & 100 & 99 & 100 & 88 & 77 & 43 \\
  \hline
  GKLS 4d Simple &  100 & 100 & 74 & 100 & 100 & 94 & 82 & 68 & 72 \\
  \hline
  GKLS 4d Hard   &  100 & 100 & 60 & 99 & 99 & 94 & 73 & 55 & 69  \\
  \hline
  GKLS 5d Simple &  100 & 100 & 86 & 100 & 100 & 98 & 100 & 88 & 82  \\
  \hline
  GKLS 5d Hard   &  100 & 100 & 77 & 100 & 93 & 79 & 86 & 77 & 78 \\
  \hline
  \end{tabular}
  \label{tab:solved}
\end{center}
\end{table}

\begin{figure}[ht]
  \centering
  \subfloat[4d Simple]{{\includegraphics[width=.5\textwidth]{images/gklss4d.pdf}}\label{fig:s4d}}
  \subfloat[4d Hard]{{\includegraphics[width=.5\textwidth]{images/gklsh4d.pdf}}\label{fig:h4d}}

  \subfloat[5d Simple]{{\includegraphics[width=.5\textwidth]{images/gklss5d.pdf}}\label{fig:s5d}}
  \subfloat[5d Hard]{{\includegraphics[width=.5\textwidth]{images/gklsh5d.pdf}}\label{fig:h5d}}
  \caption{Operating characteristics of the algorithms when solving problems from the GKLS 4d and 5d classes. Best viewed in color.}
\end{figure}

Operating characteristic of the methods (Figures \ref{fig:s4d}, \ref{fig:h4d}, \ref{fig:s5d}, \ref{fig:h5d})
demonstrates that AGS and AGS-AR faster than the other methods achieve 100\% success rate. Also on GKLS 5d Simple the DIRECT
generally has the best performance, but there are several hard problems that affect it's average number of
trials metric.

\paragraph{Robustness of AGS and AGS-AR to the Hyperparameters Choice.}

In order to investigate the influence of hyperparameters to the convergence speed of the AGS and AGS-AR,
experiments with the following settings were conducted on the problems from GKLS 5d Simple class:
\begin{itemize}
  \item AGS with $r=4$ (like in the Table \ref{tab:params});
  \item AGS with $r=6$;
  \item AGS-AR with parameters from the beginning of the Section \ref{sec:experiments}
  ($q=50\cdot\log_2(4)\cdot 25 = 2500$, $r_{min}=3,\:r_{max}=2\cdot r_{min}$);
  \item AGS-AR with $r_{max}=8$ and other parameters from the previous experiment;
  \item AGS-AR with $q=1000$ and other parameters from the beginning of the Section \ref{sec:experiments};
\end{itemize}

\begin{figure}[ht]
  \centering
  \includegraphics[width=.6\textwidth]{images/ar_stab.pdf}
  \caption{Operating characteristics of AGS and AGS-AR with different hyperparameters
  when solving problems from the GKLS 5d Simple classe. Best viewed in color.}
  \label{fig:stability}
\end{figure}

The operating characteristics collected in the experiments described above are shown in the Figure \ref{fig:stability}.
AGS with $r=6$ (the cyan-colored curve) shows the worst convergence speed, which indicates that AGS is very sensitive to choice of $r$.
Since on the start AGS-AR has the same value of $r$ as AGS with $r=6$, operating characteristics of these methods
are identical up to $K=2500$. After that point AGS-AR switches to $r=3$ and rapidly begins to increase the amount of solved problems
until the next exploration phase on $K=5000$. Intervals where AGS-AR works with $r=r_{max}$ are visible on the operating characteristics as plateaus.
Variations of $r$ and $q$ didn't drastically change the operating characteristic of AGS-AR. The latter observation shows robustness of the proposed
AGS modification with the alternating parameter $r$.

\section{Conclusions}

In the present paper, several global optimization algorithms were considered.
A comparison of efficiencies of these ones has been done on a set of test problems.
Also a scheme of hyperparameters control for the AGS algorithms was proposed and evaluated.
The results presented in this work allow making the following conclusions:
\begin{itemize}
  \item the proposed modification of the stock AGS, AGS-AR allows to pay less attention to initial hyperparameter tuning and
  performs on-par with properly tuned AGS.
  \item AGS-AR method has demonstrated the convergence
  speed and reliability at the level of DIRECT and exceeds many other algorithms, the open-source
  implementations of which are available;
  \item the stochastic optimization methods inferior to the deterministic ones in the convergence
speed and in reliability. It is manifested especially strongly on more complex multiextremal
problems;
\end{itemize}

% ---- Bibliography ----
%
\bibliographystyle{splncs04}
\bibliography{bibliography}{}
%

\end{document}
