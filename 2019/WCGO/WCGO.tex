% This is samplepaper.tex, a sample chapter demonstrating the
% LLNCS macro package for Springer Computer Science proceedings;
% Version 2.20 of 2017/10/04
%
\documentclass[runningheads]{llncs}
%
\usepackage{graphicx}

%%%%ДОБАВИЛ ДЛЯ РУССКОГО ТЕКСТА
\usepackage[utf8x]{inputenc}
\usepackage[english,russian]{babel}
\usepackage{cmap}
%%%%


% If you use the hyperref package, please uncomment the following line
% to display URLs in blue roman font according to Springer's eBook style:
% \renewcommand\UrlFont{\color{blue}\rmfamily}

\begin{document}
%
\title{Generalized dimensionality reduction scheme for global optimization 
algorithms
\thanks{This study was supported by the Russian Science Foundation, project 
No.\,16-11-10150.}}
%
\titlerunning{Generalized dimensionality reduction scheme}
% If the paper title is too long for the running head, you can set 
% an abbreviated paper title here
%
\author{Konstantin Barkalov %\orcidID{0000-0001-5273-2471}  
\and Ilya Lebedev %\orcidID{0000-0002-8736-0652}
}
%
\authorrunning{K. Barkalov, I. Lebedev}
% First names are abbreviated in the running head.
% If there are more than two authors, 'et al.' is used.
%
\institute{Lobachevsky State University of Nizhni Novgorod, Nizhni Novgorod, 
Russia \email{konstantin.barkalov@itmm.unn.ru}}
%
\maketitle              % typeset the header of the contribution
%
\begin{abstract}
\Russian
В статье рассматриваются задачи многомерной многоэкстремальной оптимизации и численные методы их решения. Об оптимизируемой функции делается лишь общее предположение, что она удовлетворяет Lipschitz condition с неизвестной априори константой. Задачи такого типа часто встречаются в приложениях. 
Рассмотрено два подхода к редукции размерности в задачах многомерной оптимизации. Первый из них использует Peano-type space-filling curves, отображающие одномерный отрезок на многомерную область. Второй основан на nested optimization scheme, that reduces the mul- tidimensional problem to a family of one-dimensional subproblems.
Предложена обобщенная схема, комбинирующая эти два подхода. В новой схеме решение многомерной задачи сводится к решению семейства задач меньшей размерности, в которых, в свою очередь, используются space-filling curves. Реализован адаптивный алгоритм, в котором все возникающие подзадачи решаются одновременно. 
Проведены численные эксперименты на нескольких сотнях тестовых задач, подтверждающие эффективность предложенной обобщенной схемы.

\keywords{Global optimization \and Multiextremal functions \and 
Dimensionality reduction \and Peano curve \and Nested optimization \and 
Numerical methods.}
\end{abstract}
%
%
%
\section{Introduction}
This paper considers ``black-box'' global optimization problems of the 
following form:
\begin{eqnarray}\label{main_problem}
& \varphi(y^\ast)=\min{\left\{\varphi(y):y\in D\right\}},\\
& D=\left\{y\in R^N: a_i\leq y_i \leq b_i, 1\leq i \leq N\right\}. \nonumber
\end{eqnarray}
The objective function is assumed to satisfy the Lipschitz condition 
\[
\left|\varphi(y')-\varphi(y'')\right|\leq L\left\|y'-y''\right\|,\; y',y'' \in
 D,\; 0<L<\infty,
\]
with the constant $L$ unknown a priori.

\Russian
Известным способом решения многоэкстремальных задач являются мультистартовые схемы. В этих схемах в области поиска размещается некоторая сетка, точки которой использутся как стартовые поиска экстремума локальным методом, а затем выбирается наименьший из найденных экстремумов. Отдельной проблемой в рамках данного подхода является выбор стартовых точек, осуществляемый, как правило, на основе метода Монте-Карло \cite{Zhigljavsky2008}. Этот подход вполне применим для задач с небольшим числом локальных минимумов, имеющих широкие области притяжения, но для задач с существенной многоэкстремальностью его эффективность резко падает. 

В настоящее время для решения задач глобальной оптимизации широко используются генетические алгоритмы (см., например, \cite{Yang2013}), которые так или иначе основаны на идеях случайного поиска. В силу простоты реализации и использования они получили большую популярность, однако по качеству работы (численной оценкой которого может служить число корректно решенных задач из некоторого тестового набора) они существенно уступают детерминированным алгоритмам \cite{Kvasov2018,Sergeyev2018}.

Если говорить о детерминированных методах глобальной оптимизации, то многие методы данного класса основаны на различных способах разбиения области поиска на систему подобластей и последующего выбора наиболее перспективной подобласти для размещения очередного испытания (вычисления целевой функции). Результаты в данном направлении представлены в работах \cite{Zilinskas2010,Paulavicius2016,Evtushenko2013,Jones2009,Sergeyev2015}.

Наконец, для разработки методов многомерной оптимизации широко применяется подход, связанный с редукцией многомерной задачи к эквивалентной одномерной или системе одномерных подзадач и последующим решением одномерных задач эффективными методами оптимизации функций одной переменной. Используются две такие схемы: редукция на основе кривых, заполняющих пространство (кривых Пеано, или разверток) \cite{Strongin2000,Sergeyev2013}, и схема рекурсивной вложенной оптимизации \cite{Strongin2000,Grishagin2001}. В статье \cite{Grishagin2016} предложена адаптивная схема редукции, обобщающая классическую схему рекурсивной оптимизации. Адаптивная схема существенно повышает эффективность оптимизации по сравнению с базовым прототипом \cite{Grishagin2016_1}. В данной работе предлагается обобщение адаптивной схемы редукции размерности, комбинирующее использование вложенной оптимизации и кривых Пеано. При таком подходе вложенные подзадачи в адаптивной схеме могут быть как одномерными, так и многомерными. В последнем случае для редукции размерности вложенных подзадач используются развертки.


\section{The global search algorithm}
\label{SectionCore}

As a core problem we consider a one-dimensional multiextremal optimization 
problem
\begin{equation}\label{uni_problem}
\varphi^\ast = \varphi(x^\ast)=\min{\left\{\varphi(x):x\in \left[0,1\right] 
\right\}}
\end{equation}
with objective function satisfying the Lipschitz condition. 

Let us give the description of the global search algorithm (GSA) applied for 
solving the above problem (according to \cite{Strongin2000}). 
GSA involves constructing a sequence of points $x^i$, where the values of the 
objective function $z^i=\varphi(x^i)$ are calculated. Let us call the 
function value calculation process the \textit{trial}. 
According to the algorithm, the first two trials are executed at the ends of 
the interval  $[a,b]$, i.e., $x^0=a,\;x^1=b$. The function values $z^0=\varphi
(x^0),\;z^1=\varphi(x^1)$  are computed and the number $k$ is set to 1. In 
order to select the point of a new trial $x^{k+1}, k\geq 1,$  it is necessary 
to perform the following steps.

\textbf{Step 1.} Renumber by subscripts (beginning from zero) the points $x^i,
\:0\leq i\leq k$, of the previous trials in increasing order, i.e.,
\[
0=x_0<x_1<\ldots <x_{k}=1.
\] 
Juxtapose to the points $x_i, 0\leq i\leq k$,  the function values $z_i=
\varphi(x_i), 0\leq i\leq k$.

\textbf{Step 2.} Compute the maximum absolute value of the first divided 
differences 
\begin{equation}\label{mu}
\mu=\max_{1\leq i\leq k}\frac{\left|z_i-z_{i-1}\right|}{\Delta_i}
\end{equation}
where $\Delta_i = x_i-x_{i-1}$. If the above formula yields a zero value, 
assume that $\mu = 1$.

\textbf{Step 3.} For each interval $(x_{i-1},x_i),1\leq i\leq k$,  calculate 
the characteristic
\begin{equation}\label{R}
R(i)=r\mu\Delta_i+\frac{(z_i-z_{i-1})^2}{r\mu\Delta_i}-2(z_i+z_{i-1}),
\end{equation} 
where $r>1$ is a predefined parameter of the method. 

\textbf{Step 4.} Find the interval $(x_{t-1},x_t)$ with the maximum 
characteristic
\begin{equation}\label{MaxR}
R(t)=\max_{1\leq i\leq {k}}R(i).
\end{equation}  

\textbf{Step 5.} Execute the new trial at the point 
\begin{equation}\label{xk1}
x^{k+1}=\frac{1}{2}(x_{t-1}+x_t) - \frac{z_t-z_{t-1}}{2r\mu}.
\end{equation}

The algorithm terminates if the condition $\Delta_t<\epsilon$ is satisfied; 
here $t$ is from (\ref{MaxR}) and $\epsilon>0$ is the preset accuracy. For 
estimation of the global solution values
\[
z_k^\ast=\min_{0\leq i \leq k}\varphi(x^i), \ x_k^\ast=\arg \min_{0\leq i \leq
 k}\varphi(x^i).
\]
are selected. 
The theory of algorithm convergence is presented in \cite{Strongin2000}.

\section{Dimensionality reduction}
\subsection{Dimensionality reduction using Peano-type space-filling curves}
\label{SectionPeano}

The use of Peano curve $y(x)$ 
\begin{equation}
\left\{y\in R^N: -2^{-1}\leq y_i \leq 2^{-1}, 1 \leq i \leq N\right\}=\left\{
y(x):0\leq x \leq 1 \right\}
\end{equation}
unambiguously mapping the interval of real axis $[0,1]$ onto a 
$N$-dimensional cube is the first of the dimension reduction methods considered.
Problems of numerical construction of Peano-type space-filling curves and the 
corresponding theory are considered in detail in \cite{Strongin2000}, 
\cite{Sergeyev2013}. Here we will note that a numerically 
constructed curve (\textit{evolvent}) is $2^{-m}$ accurate approximation of 
the theoretical Peano curve, where $m$ is an evolvent construction parameter.

By using this kind of mapping it is possible to reduce the multidimensional 
problem~(\ref{main_problem}) to a univariate problem
\begin{equation}
\varphi(y^\ast)=\varphi(y(x^\ast))=\min{\left\{\varphi(y(x)): x\in[0,1]
\right\}}.
\end{equation}
An important property of such mapping is preservation of boundedness of 
function relative differences (see \cite{Strongin2000,Sergeyev2013}). If the 
function $\varphi(y)$ in the domain $D$ satisfies the Lipschitz condition, 
then the function $\varphi(y(x))$ on the interval $[0,1]$ will satisfy a 
uniform H{\"o}lder condition
\begin{equation}\label{Holder}
\left|\varphi(y(x_1))-\varphi(y(x_2))\right|\leq H\left|x_1-x_2\right|^{1/N},
\end{equation}
where the H{\"o}lder constant $H$ is linked to the Lipschitz constant $L$ by 
the relation
\begin{equation}
H=2L\sqrt{N+3}.
\end{equation}
Condition (\ref{Holder}) allows adopting the algorithm for solving the 
one-dimensional problems presented in Section \ref{SectionCore} for solving the 
multidimensional problems reduced to the one-dimensional ones. For this, the 
lengths of intervals $\Delta_i$  involved into rules (\ref{mu}),(\ref{R}) of the
algorithm are substituted by the lengths 
\begin{equation}
\Delta_i=\left(x_i-x_{i-1}\right)^{1/N}
\end{equation}
and the following expression is introduced instead of formula (\ref{xk1}):
\begin{equation}
x^{k+1} = \frac{x_t+x_{t-1}}{2} - \mathrm{sign}(z_t-z_{t-1})\frac{1}{2r}
\left[\frac{\left|z_t-z_{t-1}\right|}{\mu}\right]^N.
\end{equation}

\subsection{Nested optimization scheme}

The nested optimization scheme of dimensionality reduction is based on the 
well-known relation (see, e.g., \cite{Carr1964})
\begin{equation}\label{nested}
\min_{y \in D}\varphi(y) = \min_{a_1\leq y_1 \leq b_1}\min_{a_2\leq y_2 \leq 
b_2}...\min_{a_N\leq y_N \leq b_N}\varphi(y),
\end{equation}
which allows replacing the solving of multidimensional problem 
(\ref{main_problem}) by solving a family of one-dimensional subproblems 
related to each other recursively.

In order to describe the scheme let us introduce a set of reduced functions 
as follows:
\begin{equation}\label{nested_N}
\varphi^N(y_1,...,y_N) = \varphi(y_1,...,y_N),
\end{equation}
\begin{equation}\label{nested_i}
\varphi^i(y_1,...,y_i) = \min_{a_{i+1}\leq y_{i+1}\leq b_{i+1}} \varphi^{i+1}(
y_1,...,y_i,y_{i+1}), 1\leq i\leq N-1.
\end{equation}

Then, according to relation (\ref{nested}), solving of multidimensional 
problem (\ref{main_problem}) is reduced to solving a one-dimensional problem 
\begin{equation}\label{nested_1}
\varphi^* = \min_{a_1\leq y_1\leq b_1}\varphi^1(y_1).
\end{equation}
But in order to evaluate the function $\varphi^1$ at a fixed point $y_1$ it 
is necessary to solve the one-dimensional problem of the second level
\begin{equation}
\varphi^1(y_1) = \min_{a_2\leq y_2\leq b_2}\varphi^2(y_1,y_2),
\end{equation}
and so on up to the univariate minimization of the function $\varphi^N(y_1
,...,y_N)$ with fixed coordinates $y_1,...,y_{N-1}$ at the $N$-th level of 
recursion.

For the nested scheme presented above, a generalization (\textit{block nested 
optimization scheme}), which combines the use of evolvents and the nested 
scheme has been elaborated in \cite{Barkalov2016}.

Let us consider vector $y$ as a vector of block variables
\begin{equation}
y=(y_1,y_2,...,y_N)=(u_1,u_2,...,u_M),
\end{equation}
where the $i$-th block variable $u_i$ is a vector of vector $y$ components, 
taken serially i.e. $u_1=(y_1,y_2,...,y_{N_1})$, $u_2=(y_{N_1+1},y_{N_1+2}
,...,y_{N_1+N_2})$,..., $u_M=(y_{N-N_M+1},y_{N-N_M+2},...,y_{N})$, where $N_1+
N_2+...+N_M=N$.

Using the new variables, main relation of the nested scheme (\ref{nested}) 
can be rewritten in the form 
\begin{equation}\label{block_nested}
\min_{y \in D}\varphi(y) = \min_{u_1\in D_1}\min_{u_2\in D_2}...\min_{u_M \in 
D_M}\varphi(y),
\end{equation}
where the subdomains $D_i, 1 \leq i \leq M$, are projections of initial 
search domain $D$ onto the subspaces corresponding to the variables $u_i, 1 
\leq i \leq M$.

The formulae defining the method of solving of problem (\ref{main_problem}) 
based on relation (\ref{block_nested}), in general, are the same to the ones 
of nested scheme (\ref{nested_N})--(\ref{nested_1}). It is only necessary to 
substitute the original variables $y_i, 1\leq i \leq N$,  by the block 
variables $u_i, 1\leq i \leq M$. 

At that, the nested subproblems 
\begin{equation}\label{block_nested_i}
\varphi^i(u_1,...,u_i) = \min_{u^{i+1}\in D_{i+1}} \varphi_{i+1}(u_1,...,u_i,u
_{i+1}), 1\leq i\leq M-1.
\end{equation}
in the block scheme are the multidimensional ones. The dimension reduction 
method based on Peano curves can be applied for solving these ones. It is a 
principal difference from the initial nested scheme. 

\subsection{Block adaptive optimization scheme}
\Russian
Решение возникающего множества подзадач вида (\ref{nested_i}) (для nested optimization scheme) или вида (\ref{block_nested_i}) (для block nested optimization scheme) может быть организовано различными способами. 
Очевидный способ (детально проработанный в \cite{Grishagin2001,Grishagin2015} для nested optimization scheme и в \cite{Barkalov2014,Barkalov2016} для block nested optimization scheme) основан на решении поздадач в соответствии с рекурсивным порядком их порождения. Однако здесь возникает потеря значительной части информации о objective function при решении многомерной задачи. Иным  подходом является адаптивная схема, в которой все подзадачи решаются одновременно, что позволяет более полно учитывать информацию о многомерной задаче и за счет этого ускорять процесс ее решения.

Для случая одномерных поздадач данный подход был теоретически обоснован и апробирован в \cite{Grishagin2016,Grishagin2016_1,Grishagin2018}. Настоящая работа предлагает обобщение адаптивной схемы для случая многомерных подзадач. Дадим краткое описание ее основных элементов.

Пусть вложенные подзадачи вида (\ref{block_nested_i}) решаются с помощью многомерного алгоритма глобального поиска, описанного в Section \ref{SectionPeano}. Тогда каждой подзадаче (\ref{block_nested_i}) можно присвоить некоторое числовое
значение, называемое характеристикой этой задачи. К качестве такой характеристики можно взять значение $R(t)$ from (\ref{MaxR}), т.е. максимальную характеристику интервалов, сформированных в данной задаче. В соответствии с правилом вычисления характеристик (\ref{R}), чем выше значение характеристики, тем более перспективной является подзадача для продолжения поиска в ней глобального минимума исходной задачи (\ref{main_problem}). Поэтому на каждой итерации выбирается подзадача с максимальной характеристикой для проведения в ней очередного испытания. Это испытание либо вычисляет значение целевой функции $\varphi(y)$ (если выбранная подзадача принадлежала уровню $j=M$), либо порождает новые подзадачи согласно (\ref{block_nested_i}) при $j\leq M-1$. В последнем случае новые порожденные задачи добавляются к текущему множеству задач, вычисляются их характеристики и процесс повторяется. Завершение процесса оптимизации происходит, когда в корневой задаче выполняется условие остановки алгоритма, решающего эту задачу.

\section{Results of numerical experiments}

One of the well-known approaches to investigating and comparing multiextremal 
optimization algorithms is based on the application of these methods for 
solving a set of test problems generated randomly.
\Russian
The comparison of the algorithms has been carried out using the Grishagin test problems $F_{gr}$ (описание данных функций см., например, в \cite{Grishagin1994}, test function 4) and the GKLS generator \cite{Gaviano2003}.

В работах \cite{Barkalov2015,Grishagin2018} было показано, что global search algorithm как с использованием разверток, так и в сочетании с адаптивной схемой редукции размерности превосходит многие известные алгоритмы глобальной оптимизации, включая методы DIRECT и DIRECT\textit{l}. Поэтому в данном исследовании мы ограничимся сравнением вариантов GSA с различными схемами редукции размерности.

Для сравнения эффективности работы алгоритмов используются два критерия: the average number of trials and operating characteristic.
Операционной характеристикой алгоритма называется функция $P(k)$, определяемая как доля задач из рассматриваемой серии, для решения которых потребовалось не более чем $k$ испытаний.
The global minimizer $y^\ast$ was considered to be found, if the algorithm generated a trial point $y^k$ in the vicinity of the global minimum, i.e. $\left|y^k-y^\ast\right| < \delta \left\|b-a\right\|$, where $\delta = 0.01$, $a$ and $b$ are the boundaries of the search domain $D$.

Первая серия экспериментов была проведена на двумерных задачах классов $F_{gr}$, GKLS \textit{Simple}, GKLS \textit{Hard} (100 функций каждого класса). В Table~\ref{tab1}  представлено среднее число испытаний, выполненных GSA с использованием evolvents ($K_e$), nested optimization scheme ($K_n$), adaptive nested optimization scheme ($K_a$). На Fig.~\ref{fig1}, Fig.~\ref{fig2}(a,b) приведены operating characteristics алгоритмов, полученные на классах $F_{gr}$, GKLS \textit{Simple}, GKLS \textit{Hard} соответственно. Непрерывная линия соответсвует алгоритму с использованием evolvents, короткий пунктир -- adaptive nested optimization scheme, длинный пунктир -- nested optimization scheme. Результаты экспериментов показывают, что GSA с использованием adaptive nested optimization scheme показывает примерно одинаковое быстродействие  по сравнению с GSA с развертками, и оба они значительно превосходят алгоритм, использующий nested optimization scheme. Поэтому в дальнейших экспериментахмы ограничимся сравнением различных вариантов адаптивной схемы редукции размерности.

\begin{table}
\centering
\caption{Average number of iterations for 2D problems.}\label{tab1}
\begin{tabular}{lccc}
\hline\noalign{\smallskip}
 &  $F_{gr}$  &  GKLS \textit{Simple} &  GKLS \textit{Hard} \\
\noalign{\smallskip}\hline\noalign{\smallskip}
 $K_e$ & 180  & 252 & 674 \\
 $K_n$ & 341  & 697 & 1252 \\
 $K_a$ & 215  & 279 & 815 \\
\noalign{\smallskip}\hline
\end{tabular}
\end{table}

\begin{figure}
\centering
\includegraphics[width=0.50\textwidth]{2D.pdf}
\caption{Operating characteristics using $F_{gr}$ class} 
\label{fig1}
\end{figure}

\begin{figure}
\begin{minipage}{0.5\linewidth}
\center{\includegraphics[width=1.0\linewidth]{2DSimple.pdf} \\ (a)}
\end{minipage}
\begin{minipage}{0.5\linewidth}
\center{\includegraphics[width=1.0\linewidth]{2DHard.pdf} \\ (b)}
\end{minipage}
\caption{Operating characteristics using 2d GKLS \textit{Simple} (a) and \textit{Hard} (b) classes}
\label{fig2}
\end{figure}

Вторая серия экспериментов была проведена на четырехмерных задачах классов GKLS \textit{Simple} and GKLS \textit{Hard} (100 функций каждого класса). В Table~\ref{tab2}  представлено среднее число испытаний, выполненных GSA с использованием adaptive nested optimization scheme ($K_a$), block adaptive nested optimization scheme ($K_ba$) с формированием двух уровней подзадач одинаковой размерности $N_1=N_2=2$. Отметим, что при использовании исходного варианта адаптивной схемы при решении задачи размерности $N=4$ формируется четыре уровня одномерных подзадач, что усложняет их обработку.

На Fig.~\ref{fig3}(a,b) приведены operating characteristics алгоритмов, полученные на классах GKLS \textit{Simple} and GKLS \textit{Hard} соответственно. Пунктирная линия соответсвует алгоритму с использованием adaptive nested optimization scheme, непрерывная --  block adaptive nested optimization scheme. 
Результаты экспериментов показывают, что использование block adaptive nested optimization scheme дает ощутимый выигрыш по числу trials (до 35\%) по сравнению с исходной adaptive nested optimization scheme.

\begin{table}
\centering
\caption{Average number of iterations for 4D problems.}\label{tab2}
\begin{tabular}{lcc}
\hline\noalign{\smallskip}
 &    GKLS \textit{Simple} &  GKLS \textit{Hard} \\
\noalign{\smallskip}\hline\noalign{\smallskip}
 $K_a$  &  21747 & 35633 \\
 $K_{ba}$ &  15942 & 33206 \\
\noalign{\smallskip}\hline
\end{tabular}
\end{table}

\begin{figure}
\begin{minipage}{0.5\linewidth}
\center{\includegraphics[width=1.0\linewidth]{4DSimple.pdf} \\ (a)}
\end{minipage}
\begin{minipage}{0.5\linewidth}
\center{\includegraphics[width=1.0\linewidth]{4DHard.pdf} \\ (b)}
\end{minipage}
\caption{Operating characteristics using 4d GKLS \textit{Simple} (a) and \textit{Hard} (b) classes}
\label{fig3}
\end{figure}


\section{Conclusion}
\Russian
В данной работе предложена обобщенная адаптивная схема редукции размерности для задач глобальной оптимизации, комбинирующая использование кривых Пеано и схему вложенной (рекурсивной) оптимизации. Для решения редуцированных подзадач меньшей размерности используется алгоритм глобального поиска. Приведена вычислительная схема алгоритма, рассмотрены основные вопросы, связанные с использованием адаптивной схемы редукции размерности.
Проведены вычислительные эксперименты на серии тестовых задач с целью сравнения эффективности различных схем редукции размерности. 
Результаты экспериментов показывают, что использование block adaptive nested optimization scheme может значительно сократить число испытаний, необходимое для решения задачи с заданной точностью. 
Дальнейшие работы в развитии методов решения задач глобальной оптимизации, основанных на данном способе редукции размерности, могут быть связаны с использованием локальных оценок константы Липшица в различных подзадачах; схемы локального оценивания подробно рассмотрены, например, в \cite{Grishagin2015,Sergeyev2016}.


%
% ---- Bibliography ----
%
% BibTeX users should specify bibliography style 'splncs04'.
% References will then be sorted and formatted in the correct style.
%
 \bibliographystyle{splncs04}
 \bibliography{bibliography}


\end{document}
