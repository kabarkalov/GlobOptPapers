% Latex-template for one-page abstract submissions.
% NUMTA2019 - Numerical Computations: Theory and Algorithms
% 15-21 June 2019, Le Castella - Isola Capo Rizzuto, Crotone (Italy)

\documentclass[oribibl]{llncs}

% Please, do not use your own macros/redefinitions.

\usepackage{fancyhdr}
\renewcommand{\headrulewidth}{0pt}
\renewcommand{\footrulewidth}{0pt}

\usepackage[utf8x]{inputenc}
\usepackage[english,russian]{babel}
\usepackage{cmap}


\begin{document}
\cleardoublepage

\title{A new way to speedup in global optimization problems}
\author{Roman Strongin, Konstantin Barkalov, Semen Bevzuk}

%%% Please note that the first author is supposed to be the speaker.

\institute{University of Nizhni Novgorod, Nizhni Novgorod, Russia\\
\email{barkalov@vmk.unn.ru}}

\maketitle

\thispagestyle{fancy}

\textbf{Keywords.} Global optimization; multiextremal problems; Lipschitz constant.

 \vspace*{0.5cm}

%-----------------------------------------------------------------

В работе рассматриваются задачи глобальной оптимизации c black-box objective function, удовлетворяющей условию Липшица. Важной проблемой в задачах данного класса является адекватная оценка априори неизвестной константы Липшица на основе результатов проведенных search trials. В монографии \cite{Book} предложены различные подходы к учету как глобальных, так и локальных свойств objective function. В частности, показали свою перспективность  алгоритмы с использованием локальных оценок константы Липшица \cite{Sergeyev10, Sergeyev16}.

В данной статье предлагается новый способ учета локальных свойств objective function. Предложенный способ основан на использовании в информационно-статистическом алгоритме глобального поиска \cite{Book} одновременно двух оценок константы Липшица: нижней и верхней. Правила нового алгоритма обеспечивают автоматическое переключение между указанными двумя оценками в зависимости от поведения оптимизируемой функции и от стадии поиска.

Предложенный новый алгоритм гарантирует сходимость к global minimizer. Результаты решения серии тестовых задач разной размерности подтверждают как наличие глобальной сходимости  нового алгоритма, так и его ускорение по сравнению с исходным алгоритмом.

\textbf{Acknowledgements.}

This research was supported by the the Russian Science Foundation, project No.\,16-11-10150.

\vspace{0.5cm}

\begin{thebibliography}{4}

\bibitem{Book} Strongin R.G., Sergeyev Ya.D. (2000) \emph{Global optimization with non-convex constraints. Sequential and parallel algorithms}. Kluwer Academic Publishers, Dordrecht.

\bibitem{Sergeyev10} Lera D., Sergeyev Ya.D. (2010)  (2010) An information global minimization algorithm using the local improvement technique. \emph{J. Glob. Optim.}, Vol.~48(1), pp.~99--112.

\bibitem{Sergeyev16} Sergeyev Ya.D., Mukhametzhanov M.S., Kvasov D.E., Lera D. (2016) Derivative-free local tuning and local improvement techniques embedded in the univariate global optimization. \emph{J. Optim. Theory Appl.}, Vol.~171(1), pp.~186--208.

%\bibitem{Strongin18} Barkalov K., Strongin R. (2018) Solving a set of global optimization problems by the parallel technique with uniform convergence. \emph{J. Glob. Optim.}, Vol.~71(1), pp.~21--36.


\end{thebibliography}

\end{document}
