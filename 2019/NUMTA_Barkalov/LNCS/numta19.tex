% This is samplepaper.tex, a sample chapter demonstrating the
% LLNCS macro package for Springer Computer Science proceedings;
% Version 2.20 of 2017/10/04
%
\documentclass[runningheads]{llncs}
%
\usepackage{graphicx}
\usepackage{hyperref}
\usepackage{amsmath}

% Used for displaying a sample figure. If possible, figure files should
% be included in EPS format.
%
% If you use the hyperref package, please uncomment the following line
% to display URLs in blue roman font according to Springer's eBook style:
% \renewcommand\UrlFont{\color{blue}\rmfamily}

\usepackage[utf8x]{inputenc}
\usepackage[english,russian]{babel}


\begin{document}
%
\title{Acceleration of Global Search through Dual Lipschitz Constant Estimates
\thanks{This research was supported by the the Russian Science Foundation,
project No.\,16-11-10150.}}
%
%\titlerunning{Abbreviated paper title}
% If the paper title is too long for the running head, you can set
% an abbreviated paper title here
%
\author{Roman Strongin%\orcidID{0000-0003-0390-6695} 
\and Konstantin Barkalov%\orcidID{0000-0001-5273-2471} 
\and Semen Bevzuk%\orcidID{0000-0002-3845-5356}
}
%
\authorrunning{R. Strongin et al.}
% First names are abbreviated in the running head.
% If there are more than two authors, 'et al.' is used.
%
\institute{
Lobachevsky State University of Nizhni Novgorod, Nizhni Novgorod, Russia
\email{konstantin.barkalov@itmm.unn.ru}
}
%
\maketitle              % typeset the header of the contribution
%
\begin{abstract}
The paper considers global optimization problems with a black-box objective 
function satisfying the Lipschitz condition. Efficient algorithms for this 
class of problems require reliable estimates of the Lipschitz constant to be 
introduced. The various approaches have been proposed to take into account both
global and local properties of the objective function. In particular, algorithms
using local estimates of the Lipschitz constant have shown their potential.
The new approach presented in this paper is based on simultaneous use of two
estimates: one is substantially larger then the other. 
The larger estimate ensures global convergence and the smaller one reduces 
the total number of trials needed to find the global optimizer.
Results of numerical experiments on the random sample of multidimensional 
functions demonstrate the efficiency of the suggested approach.  

\keywords{Global optimization \and Multiextremal problems 
\and Lipschitz constant estimates}
\end{abstract}
%
%
%
\section{Introduction}

The paper considers global optimization problems of the form 
\begin{gather}
 \varphi(y^\ast)=\min{\left\{\varphi(y):y\in D\right\}},\\
 D=\left\{y\in R^N: a_i\leq y_i \leq b_i, 1\leq i \leq N\right\},
\end{gather}
where the objective function is a black-box function and assumed to satisfy the Lipschitz condition
\[
\left|\varphi(y_1)-\varphi(y_2)\right|\leq L\left\|y_1-y_2\right\|,\; y_1,y_2 \in D,
\]
with the constant $L$ unknown a priori.

Предположение липшицевости целевой функции является типичным для многих подходов к разработке алгоритмов глобальной оптимизации (ссылки). 

%Здесь же - про редукцию размерности!
% Или в конце введения? 
% Или вообще в следующем разделе?

При этом наиболее важной проблемой, так или иначе решаемой в данных алгоритмах, является адаптивная оценка неизвестной константы Липшица на основе полученной поисковой информации. 

Значение константы Липшица существенно влияет на скорость сходимости алгоритмов липшицевой глобальной оптимизации, поэтому столь важным является вопрос ее корректной оценки. Заниженная оценка инстинного значения константы Липшица может привести к потере сходимости алгоритма к глобальному решению. В то же время слишком большое значение оценки константы $L$ для целевой функции, %предполагает сложную структуру функции с резкими перепадами ее значений. Поэтому завышенная оценка константы L, 
не соответствующее ее истинному поведению, влечет за собой медленную сходимость алгоритма к точке глобального минимума. 

Известны несколько типичных способов адаптивного оценивания константы Липшица:
\begin{itemize}
	\item оценивание глобальной константы Липшица во всей области поиска D (см., например, [] ).
	\item оценивание локальных оценок констант Липшица $L_i$ в различных подобластях $D_i$ области поиска $D$ (см. ).
	\item выбор оценок константы $L$ из некоторого множества возможных значений.
\end{itemize}

%связка
Каждый из указанных подходов обладает своими достоинствами и недостатками. Например, 

В данной работе предлагается алгоритм, который испольует две глобальных оценки константы Липшица, одна из которых значительно больше другой. 
The larger estimate ensures global convergence and the smaller one reduces the total number of trials needed to find the global optimizer.

% геометрическая интерпретация

Выбор того, какая из двух оценок будет использоваться в правилах алгоритма, осуществляется адаптивно, в зависимости от поведения функции и фазы поиска.

Строгое обоснование предложенного подхода выходит за рамки данной первоначальной публикации и будет сделано в последующих работах. 
Здесь же приведены результаты вычислительных экспериментов, наглядно демонстрирующие эффективность предложенного алгоритма.
При проведении вычислительных экспериментов было решено несколько сотен тестовых многоэкстремальных задач разной размерности.
 
% Константа Липшица - основное в методах глобальной оптимизации


\section{Global search algorithm}




According to the global search algorithm, the first two trials are executed at 
the points $y^0=y(0), y^1=y(1)$. The choice of the point $y^{k+1},k\geq 1,$  
for the next $(k+1)$-th trial is defined by the following rules.

\begin{enumerate}
	\item 
	Renumber the preimages of all the points $y^i=y(x^i)$
	from the trials already performed  	
%\begin{equation}\label{y_i} 
%y^0=y(x^0), y^1=y(x^1),...,y^k=y(x^k)
%\end{equation}
by subscripts in the increasing order of their coordinates, i.e.
\begin{equation}\label{x_i}
0=x_0<x_1<\dots <x_k=1,
\end{equation}
and associate these with the values $z_i=\varphi(y(x_i)), 0\leq i \leq k,$ 
computed at these points.
\item
Compute the maximum absolute value of the first divided differences
\begin{equation}\label{mu}
\mu = \max_{1 \leq i \leq k}\frac{\left|z_i-z_{i-1}\right|}{\Delta_i},
\end{equation}
where $\Delta_i=\left(x_i-x_{i-1}\right)^{1/N}$. If $\mu = 0$, set $\mu = 1$.
\item
For each interval $(x_{i-1}, x_i), \; 1\leq i \leq k,$  calculate the value
\begin{equation}\label{R}
R(i)=\Delta_i+\frac{(z_i-z_{i-1})^2}{r^2\mu^2\Delta_i}-2\frac{z_i+z_{i-1}-z^*}{r\mu}
\end{equation}
called the \textit{characteristic} of the interval; the real number $r>1$ being 
the input parameter of the algorithm.
\item 
Select the interval $(x_{t-1},x_t)$ corresponding to the maximum characteristic
\begin{equation}\label{MaxR}
R(t)= \max_{1 \leq i \leq k}R(i).
\end{equation}
\item
Carry out the next trial at the point $x^{k+1}\in(x_{t-1},x_t)$ calculated using
the following formula
\begin{equation}\label{xk1}
x^{k+1} = \frac{x_t+x_{t-1}}{2} - \mathrm{sign}(z_t-z_{t-1})\frac{1}{2r}
\left[\frac{\left|z_t-z_{t-1}\right|}{\mu}\right]^N.
\end{equation}
\end{enumerate}

The algorithm terminates if the condition $\Delta_t < \epsilon$ is satisfied
where $t$ is from (\ref{MaxR}), and $\epsilon>0$ is the predefined accuracy. 

The theory of convergence of this algorithm is provided in \cite{Strongin2000}.
%The modifications taking into account .... are given in [....].

\section{Алгоритм с двойной оценкой Lipschitz constant}

Алгоритм глобального поиска, изложенной в предыдущем разделе, предназначен для решения многоэкстремальных задач, в которых целевая функция удовлетворяет условию Липшица. При этом для для работы алгоритма не требуется задания  значений константы Липшица. Оценка константы осуществляется в процессе глобального поиска на основе имеющейся поисковой информации. 
В соответствии с теоремой \cite{Strongin2000}, последовательность точек испытаний $\{y^k\}$ будет сходиться к global minimizer $y^*$, если выполнено условие  
\begin{equation}\label{cond}
r\mu > 2^{3-1/N}L\sqrt{N+3}.
\end{equation}
Таким образом, подходящий выбор параметра $r$ из (\ref{R}) позволяет использовать значение $r\mu / 2^{3-1/N}\sqrt{N+3}$ как оценку константы Липшица для целевой функции $\varphi(y)$.


Выполнение условия (\ref{cond}) будет гарантировано при выборе большого значения параметра $r$, однако при этом метод выполнит большое количество испытаний до выполнения условия остановки.
Выбор малого значения параметра $r$ (что соответствует заниженной оценке константы Липшица) значительно сократит число испытаний, но может нарушить сходимость к глобальному экстремуму.

Перспективным является подход, при котором в правилах алгоритма будут использованы две оценки константы Липшица, одна из которых будет значительно выше другой, что соответствует использованию в алгоритме двух парметров $r_{max}$ и $r_{min}$, где $r_{max} > r_{min}$.


Характеристики $R(i)$ зависят от параметра $r$, и они будут несравнимы при разных значениях $r$.

Сравнимость характеристик можно получить, если ввести нормировочные коэффициент (формула).

Таким образом, правила алгоритма с двойной оценкой константы Липшица будут полностью повторять правила алгоритма глобального поиска, за исключением правила 3 алгоритма.

Новое правило вычисленя характеристики $R(i)$ интервала $(x_{i-1}, x_i)$ будет состоять из следующих действий:
\begin{itemize}
\item
Вычислить значение $R_{max}(i)$, соответствующее большей оценке константы Липшица
\[
R_{max}(i)=\Delta_i+\frac{(z_i-z_{i-1})^2}{r_{max}^2\mu^2\Delta_i}-2\frac{z_i+z_{i-1}-z^*}{r_{max}\mu}.
\]
\item
Вычислить значение $R_{min}(i)$, соответствующее меньшей оценке константы Липшица
\[
R_{min}(i)=\Delta_i+\frac{(z_i-z_{i-1})^2}{r_{min}^2\mu^2\Delta_i}-2\frac{z_i+z_{i-1}-z^*}{r_{min}\mu}.
\]
\item
Определить характеристику интервала $R(i)$ как
\[
R(i) = \max\{\rho R_{min}(i),R_{max}(i)\}, где \rho = \left(\frac{1-1/r_{max}}{1-1/r_{min}}\right)^2.
\]
\end{itemize}


\section{Numerical Experiments}



\section{Conclusion}



%
% ---- Bibliography ----
%
% BibTeX users should specify bibliography style 'splncs04'.
% References will then be sorted and formatted in the correct style.

\bibliographystyle{splncs04}
\bibliography{bibliography}

%\begin{thebibliography}{8}
%\end{thebibliography}
\end{document}
