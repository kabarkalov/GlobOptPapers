% This is samplepaper.tex, a sample chapter demonstrating the
% LLNCS macro package for Springer Computer Science proceedings;
% Version 2.20 of 2017/10/04
%
\documentclass[runningheads]{llncs}
%
\usepackage{graphicx}
%\usepackage{hyperref}
% Used for displaying a sample figure. If possible, figure files should
% be included in EPS format.
%
% If you use the hyperref package, please uncomment the following line
% to display URLs in blue roman font according to Springer's eBook style:
% \renewcommand\UrlFont{\color{blue}\rmfamily}

\begin{document}
%
\title{Acceleration of Global Search through Dual Lipschitz Constant Estimates
\thanks{This research was supported by the the Russian Science Foundation, project No.\,16-11-10150.}}
%
%\titlerunning{Abbreviated paper title}
% If the paper title is too long for the running head, you can set
% an abbreviated paper title here
%
\author{Roman Strongin%\orcidID{0000-0003-0390-6695} 
\and Konstantin Barkalov%\orcidID{0000-0001-5273-2471} 
\and Semen Bevzuk%\orcidID{0000-0002-3845-5356}
}
%
\authorrunning{R. Strongin et al.}
% First names are abbreviated in the running head.
% If there are more than two authors, 'et al.' is used.
%
\institute{
Lobachevsky State University of Nizhni Novgorod, Nizhni Novgorod, Russia
\email{konstantin.barkalov@itmm.unn.ru}
}
%
\maketitle              % typeset the header of the contribution
%
\begin{abstract}
The abstract should briefly summarize the contents of the paper in
150--250 words.

\keywords{First keyword  \and Second keyword \and Another keyword.}
\end{abstract}
%
%
%
\section{Introduction}

The paper considers global optimization problems with a black-box objective function satisfying the Lipschitz condition. 
Efficient algorithms for this class of problems require reliable estimates of the Lipschitz constant to be introduced. 
In fundamental work \cite{Strongin2000} various approaches have been proposed to take into account both global and local properties of the objective function. 
%В частности, показали свою перспективность  алгоритмы с использованием локальных оценок константы Липшица 
In particular, algorithms using local estimates of the Lipschitz constant have shown their potential \cite{Sergeyev2010,Sergeyev2016}.

The new approach presented in this paper is based on simultaneous use of two estimates: one is substantially larger then the other. 
The larger estimate ensures global convergence and the smaller one reduces the total number of trials needed to find the global optimizer.
Results of numerical experiments on the random sample of multidimensional functions demonstrate the efficiency of the suggested approach.  

\section{Conclusion}

%
% ---- Bibliography ----
%
% BibTeX users should specify bibliography style 'splncs04'.
% References will then be sorted and formatted in the correct style.

\bibliographystyle{splncs04}
\bibliography{bibliography}

%\begin{thebibliography}{8}
%\end{thebibliography}
\end{document}
