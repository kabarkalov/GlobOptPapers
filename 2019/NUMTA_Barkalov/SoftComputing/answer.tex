%% Based on a TeXnicCenter-Template by Gyorgy SZEIDL.
%%%%%%%%%%%%%%%%%%%%%%%%%%%%%%%%%%%%%%%%%%%%%%%%%%%%%%%%%%%%%

%------------------------------------------------------------
%
\documentclass[12pt]{article}%
%
\usepackage{amsmath}%
\usepackage{amsfonts}%
\usepackage{amssymb}%
\usepackage{graphicx}
\usepackage{xcolor}
\usepackage{ulem}
\usepackage{cancel}	
%\usepackage[utf8]{inputenc}
%\usepackage[english, russian]{babel} % Русские и английские переносы

%-------------------------------------------

\begin{document}

\section*{Answers to the reviewer's comments}

We are very grateful to the reviewer for his/her useful comments. Please find below our corrections that we suggest to clarify the issues.

\phantom{f}

\textbf{Question 1}. Page 7, Column 1, lines 36--56

... It was shown in (22) the inequality $R_{glob}(i) \geq \Delta_i\left(1-\frac{1}{r_M^{glob}} \right)^2$, 
but in (23) the equality $R_{glob}(i) = \Delta_i \left(1-\frac{1}{r_M} \right)^2$ 
is used both for $r_M = r_{glob}$ and $r_M = r_{loc}$. ... 
This is not sufficient to prove that $R_{glob}(i)\geq R_{loc}(i)$.
In order to obtain (23) and (24) $\mu_M$ has to be equal to 
\[
\mu_M = \frac{\left|z_i-z_{i-1}\right|}{\left(x_i-x_{i-1}\right)^{1/N}}
\]
if $\nu(x_{i-1})=\nu(x_i)=M$. Now can we be sure that this condition is fulfilled?

%\phantom{f}

\textbf{Answer}.
We agree that some of the formulations we used in the paper to prove that $R_{glob}(i)\geq R_{loc}(i)$ were not entirely clear.
We have reformulated the part of the proof on page 7, column 1. 
All corrections have been incorporated in the paper and highlighted in red.	

p.7, column 1, lines after formula (21).

If $\nu(x_{i-1}) = \nu(x_i) = M$, then, taking into account (12), the following inequality will be valid
$$
	z_i+z_{i-1}-2z_M^* = |z_i-z_{i-1}| = \mu \Delta_i \leq \mu_M\Delta_i,
$$
where $\mu = |z_i-z_{i-1}|/\Delta_i$.
From here, it will follow that for any value $r_M>1$ the following inequalities will hold
\begin{align}
	R(i) & = \Delta_i + \frac{(z_i-z_{i-1})^2}{r_M^2\mu_M^2\Delta_i} - 2\frac{z_i+z_{i-1}-2z_M^*}{r_M\mu_M} = \nonumber \\
	& = \Delta_i + \frac{\mu^2\Delta_i}{r_M^2\mu_M^2} - 2\frac{\mu\Delta_i}{r_M\mu_M} = \nonumber \\
  & = \Delta_i\left(1 + \left(\frac{\mu}{r_M\mu_M}\right)^2 - 2\frac{\mu}{r_M\mu_M}\right) = \nonumber \\
	& = \Delta_i\left(1 - \frac{\mu}{r_M\mu_M}\right)^2 \geq  \Delta_i \left( 1-\frac{1}{r_M}\right)^2 > 0.\nonumber
\end{align}


And because $r_{M}^{glob} > r_{M}^{loc} > 1$ and $\mu \leq \mu_M$, then
\[ 
	R_{glob}(i) = \Delta_i\left(1-\frac{\mu}{r_M^{glob}\mu_M} \right)^2 > \Delta_i\left(1-\frac{\mu}{r_M^{loc}\mu_M} \right)^2 = R_{loc}(i).
\]

\phantom{f}

\textbf{Question 2}. p. 9 lines 17--22 are not clear.

If $N = 2$ $\max_{0\leq\alpha\leq 1} {\left( \alpha^{\frac{1}{2}}+(1-\alpha)^{\frac{1}{2}} \right)} = \sqrt{2} \neq 2^{3-1/2}=2^{5/2}$ as in line 22. The maximum value is $2^{(N-1)/N}$.

%\phantom{f}

\textbf{Answer}. Yes, the maximum value is $2^{1-1/N}$.
There was an inaccuracy when typing. To correct it , we have removed the extra symbol ``4'' in the formula on p. 9, lines 17--22

\begin{align*}
	...& \geq \Delta_i\left(1-4\frac{L_{\nu}\sqrt{N+3}\max_{0\leq\alpha\leq1} {\left( \alpha^{\frac{1}{N}}+(1-\alpha)^{\frac{1}{N}} \right)}}{r_{\nu}\mu_{\nu}} \right) + 4\frac{z_{\nu}^{\ast}-g_{\nu}\left( y(\bar x) \right)}{r_{\nu}\mu_{\nu}} = \\
	& = \Delta_i\left(1- { \color{red} \cancel 4} \frac{2^{3-1/N}L_{\nu}\sqrt{N+3}}{r_{\nu}\mu_{\nu}} \right)+4\frac{z_{\nu}^{\ast}-g_{\nu}\left( y(\bar x) \right)}{r_{\nu}\mu_{\nu}} > 0,
\end{align*}
 
i.e. $4 \cdot 2^{1-1/N} = 2^{3-1/N}$.


\phantom{f}

\textbf{Question 3}. The proof is not self contained, in particular in lines 41--49 the used statements have not appeared before the proof.

%\phantom{f}

\textbf{Answer}. At the beginning of the proof of the theorem we have assumed that $x'$ is a limit point for the IA-DL and  $x' \neq \bar x$. 
This assumption implies that in the method IA-DL in the formula (25) for computing the characteristic, the second component takes the highest values, i.e. $R_{glob}(t) < \rho R_{loc}(t)$. This fact corresponds to using the original index algorithm (IA) with parameters $r_\nu = r_\nu^{loc}, \; 1 \leq \nu \leq m+1$.

In Strongin and Sergeyev (2000) it has been proven that for any limit point $x'$ of the sequence $\{x^k\}$, generated by IA while solving the problem at any $r_\nu > 1, \; 1 \leq \nu \leq m+1$, there exists a sequence
of the intervals $[x_{t-1}, x_t], t=t(k)$, contracting to the point $x'$, i.e. $x' \in [x_{t-1}, x_t], t=t(k)$. Characteristics $R(t)$ of the intervals of the specified sequence will be positive and $R(t) \rightarrow 0$ at $k \rightarrow \infty$. 
Here, we have used this statement without repeating its proof that can be found in Strongin and Sergeyev (2000), p. 523--525.

We have expanded the comment about this on page 9, column 1.

\phantom{f}

\textbf{Question 4}. Both $j$ and $t$ depend from $k$. Although it was proved that $R_{glob}(j) > 0$ for a
fixed $j$, how can we be sure that the condition $R_{glob}(j) > \rho R_{loc}(t)$ is fulfilled for a large enough value of $k$?

%\phantom{f}

\textbf{Answer}.
Yes, both $j$ and $t$ depend from $k$, as the numbering of the trial points (by the lower index) changes at every iteration and the numbering of the intervals changes respectively.

However, number $j(k)$ corresponds to the same interval with fixed characteristic $R_{glob}(j(k))>0$, where $R_{glob}(j(k))= R_{glob}(j(k+1))$, whereas number $t(k)$ corresponds to different intervals with different characteristics $R_{loc}(t(k))$, where $R_{loc}(t(k)) \rightarrow 0$. 
From this we can conclude that the condition $R_{glob}(j) > \rho R_{loc}(t)$ is fulfilled for a large enough value of $k$.

Corresponding comments have been incorporated in the paper on page 9, column 2.


\end{document}
