%% Based on a TeXnicCenter-Template by Gyorgy SZEIDL.
%%%%%%%%%%%%%%%%%%%%%%%%%%%%%%%%%%%%%%%%%%%%%%%%%%%%%%%%%%%%%

%------------------------------------------------------------
%
\documentclass[12pt]{article}%
%
\usepackage{amsmath}%
\usepackage{amsfonts}%
\usepackage{amssymb}%
\usepackage{graphicx}
\usepackage{xcolor}
\usepackage{ulem}
\usepackage{cancel}	
%\usepackage[utf8]{inputenc}
%\usepackage[english, russian]{babel} % Русские и английские переносы

%-------------------------------------------

\begin{document}

\section*{Answers to the reviewer's comments}

We are very grateful to the reviewer for his/her useful comments. Please find below our corrections that we suggest to clarify the issues.

\phantom{f}

\textbf{Question 1. Comment.} As shown in the last part of the proof for $\nu(x_{i-1}) = \nu(x_i) = M$
\[
\frac{R_{glob}(i)}{R_{loc}(i)}=\left( \frac{1-\frac{\mu}{\mu_M r_M^{glob}}}{1-\frac{\mu}{\mu_M r_M^{loc}}} \right)^2
\]
For these reasons an update of the formula (24) is required.

\textbf{Answer}. Yes, we agree, that 
\[
\frac{R_{glob}(i)}{R_{loc}(i)}=\left( \frac{1-\frac{\mu}{\mu_M r_M^{glob}}}{1-\frac{\mu}{\mu_M r_M^{loc}}} \right)^2 = \left( \frac{1-\frac{1}{r_M^{glob}}}{1-\frac{1}{r_M^{loc}}} \right)^2 = \rho
\]
only if $\mu = \mu_M$.

Accordingly, we have corrected both the formula (24) and the comment following it (see page 7, column 1).
In this comment, we explain that the value 
\[
\rho = \left( \frac{1-\frac{1}{r_M^{glob}}}{1-\frac{1}{r_M^{loc}}} \right)^2
\] 
can be considered as the maximum possible value of the equalization coefficient for $R_{loc}(i)$ and $R_{glob}(i)$.
The use of the equalization coefficient equal to $\rho$ will correspond to equalizing the largest possible difference between the values of $R_{loc}(i)$ and $R_{glob}(i)$ and will contribute to a more local nature of the search when the difference between $R_{loc}(i)$ and $R_{glob}(i)$ is not at its maximum, i.e. when $\mu < \mu_M$.

%To remove all questions мы провели вычислительные эксперименты с использованием в алгоритме выравнивающего  коэффициента $\rho '$ , и они показали худшие результаты, чем использование $\rho$ из . 

\phantom{f}

\phantom{f}

\textbf{Question 2. Comment.} Ok

\break

\textbf{Question 3. Comment.} I would suggest to state these results in a preliminary lemma in order
to insert the reference about these in the proof of Theorem 2 replacing the
comments in the lines -11/-1 on p.9 column 1 of the updated paper.

\textbf{Answer}. A corresponding lemma has been added on page 8, column 2, and the reference to it has been inserted in the proof of Theorem 2.

\phantom{f}

\textbf{Question 4. Comment.} I would suggest to explain better the equality $R_{glob}(j(k)) = R_{glob}(j(k+1))$
recalling the assumption that $x'	$ is a limit point for the IA-DL and $x' \neq \bar x$.

\textbf{Answer}. We inserted the suggested notation $\bar k$ on page 9, column 1, for the number of trials starting from which the trial points will not fall into the interval $[x_{j-1}, x_j]$. We also used this notation on page 10, column 1, to explain the inequality $R_{glob}(j) > \rho R_{loc}(t)$.

%\break
%
	%This statement can be interpreted as the incomparability of the local and global characteristics: the global characteristic of the best interval is larger if $\nu(x_{i-1})=\nu(x_i)=M$. At the same time, it follows from the formulae (\ref{R_glob_greater_R_loc}) that 
	%
%\begin{equation}\label{R_glob_equal_R_loc}
	%R_{glob}(i) = \left( \frac{1-\frac{\mu}{\mu_M r_M^{glob}}}{1-\frac{\mu}{\mu_M r_M^{loc}}} \right)^2 R_{loc}(i),
%\end{equation}
%wherein the maximum value of the equalization coefficient is 
%\begin{equation}\label{new_rho}
%\rho=\left(\frac{1-1/r_{M}^{glob}}{1-1/r_{M}^{loc}}\right)^2 
%\end{equation}
%in the case when $\mu = \mu_M$. %, where $\mu = |z_i-z_{i-1}|/\Delta_i$ and $\mu_M$ is from (\ref{current_lower_bounds}).
%The use of the equalization coefficient equal to $\rho$ from (\ref{new_rho}) in all cases will correspond to equalizing the largest possible difference between the values of $R_{loc}(i)$ and $R_{glob}(i)$ and will contribute to a more local nature of the search when the difference between $R_{loc}(i)$ and $R_{glob}(i)$ is not the largest, i.e. when $\mu < \mu_M$.
%
%%Использование нормирующего коэффициента равного $\rho$ из (\ref{new_rho})  будет соответствовать выравниванию наибольшей возможной разницы между значениями $R_{loc}(i)$ и $R_{glob}(i)$ и будет способствовать более ``локальному'' характеру поиска в случае когда разница между $R_{loc}(i)$ и $R_{glob}(i)$ не является максимальной (т.е. когда $\mu_M > \mu$). 
%
%Based on the relations (\ref{R_glob_equal_R_loc}) and (\ref{new_rho}), one can formulate an \textit{index algorithm with dual estimate of the Lipschitz constants} (hereafter referred to as IA-DL).
%

\end{document}
