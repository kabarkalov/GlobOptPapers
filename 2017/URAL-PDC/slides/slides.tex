\documentclass[aspectratio=1610]{beamer}

\usetheme{unnslides}
\usefonttheme{professionalfonts}

\usepackage{listings}
\usepackage{graphicx}
\usepackage{caption}
\usepackage{cmbright}
\usepackage{fontspec}
\usepackage{unicode-math}
\usepackage{amsfonts}

\setromanfont{CMU Serif}
\setsansfont{CMU Sans Serif}
\setmathfont{Latin Modern Math}

\usepackage{polyglossia}
%\setbeamertemplate{itemize item}{\color{black}$\blacktriangleright$}

\DeclareMathOperator*{\argmax}{arg\,max}
\DeclareMathOperator*{\argmin}{arg\,min}
\DeclareMathOperator{\sign}{sign}
\DeclareMathOperator{\re}{Re}

\graphicspath{ {../paper/images/}{img/} }
%set pages numeration
\setbeamertemplate{footline}[frame number]
\setbeamertemplate{headline}{}
\setlength\abovecaptionskip{-1pt}

\title{Parallel Multi-objective Optimization Method for Finding Complete Set of Weakly Efficient Solutions}
\author{\textbf{Vladislav~Sovrasov}}
\institute{Lobachevsky University}
\date{}

\begin{document}
\begin{frame}[noframenumbering,plain]
\titlepage
\end{frame}

\begin{frame}
  \frametitle{Problem statement}
  \begin{displaymath}
    \label{eq:problem}
    \min\{f(y): y\in D\}, D=\{y\in \mathbb{R}^n: a_i \leqslant y_i \leqslant b_i, 1\leqslant i
  \leqslant n \}.
\end{displaymath}
\end{frame}

\begin{frame}
  \frametitle{Dimension reduction}
  Peano-type curve \(y(x)\) allows to reduce dimension of the original multi-objective problem:
  \begin{displaymath}
  \lbrace y\in R^N:-2^{-1}\leqslant y_i\leqslant 2^{-1},1\leqslant i\leqslant
  N\rbrace=\{y(x):0\leqslant x\leqslant 1\}
  \end{displaymath}


  \begin{eqnarray*}
    \varphi(x)=\max\{h(x,y):y\in [0,1]\},x\in [0,1],\\
    h(x,y)=\max\{f_i(x)-f_i(y):1\leqslant i\leqslant m\}.
  \end{eqnarray*}

  \begin{displaymath}
    \varphi^*=\min\{\varphi(x):x\in [0,1]\}.
  \end{displaymath}
\end{frame}

\begin{frame}
  \frametitle{Optimization method}

\end{frame}

\begin{frame}
  \frametitle{Results}

  Number of iterations decreases when increasing number of threads \(p\).
  \begin{table}
    \centering
    \begin{tabular}{|l|p{1.5cm}|p{1.5cm}|p{1.5cm}|p{1.5cm}|p{1.5cm}|}
  \hline
  \textbf{Problem} & \multicolumn{5}{c|}{\(p\)}\\
  \cline{2-6}
    & \(1\) & \(2\) & \(4\) & \(8\) & \(16\)\\
  \hline
  Markin-Strongin & 1041(198) & 516(198) & 256(185) & 131(197) & 68(191) \\
  \hline
  Fonseca and Fleming 2d & 1181(93) & 636(99) & 386(111) & 176(95) & 106(97) \\
  \hline
  Fonseca and Fleming 3d & 5346(160) & 3551(183) & 1186(143) & 606(153) & 351(142) \\
  \hline
  Viennet problem & 4896(276) & 2156(273) & 1226(270) & 631(287) & 286(274)\\
  \hline
  Poloni's function & 3351(102) & 1706(90) & 856(88) & 426(96) & 201(99) \\
  \hline
  \end{tabular}
  \end{table}

\end{frame}

\begin{frame}
  \frametitle{Results (speedup in iterations)}
  \begin{table}
    \centering
    \begin{tabular}{|l|p{1.5cm}|p{1.5cm}|p{1.5cm}|p{1.5cm}|}
  \hline
  \textbf{Problem} & \multicolumn{4}{c|}{\(p\)}\\
  \cline{2-5}
    & \(2\) & \(4\) & \(8\) & \(16\)\\
  \hline
  Markin-Strongin & 2.02 & 4.07 & 7.95 & 15.31 \\
  \hline
  Fonseca and Fleming 2d & 1.86 & 3.06 & 6.71 & 11.14 \\
  \hline
  Fonseca and Fleming 3d & 1.51 & 4.51 & 8.82 & 15.23 \\
  \hline
  Viennet problem & 2.27 & 3.99 & 7.76 & 17.12\\
  \hline
  Poloni's problem & 1.96 & 3.91 & 7.87 & 16.67 \\
  \hline
  \end{tabular}
  \end{table}

\end{frame}

\begin{frame}
  \frametitle{Results (speedup in time)}
  \begin{table}[ht]
    \centering
    \begin{tabular}{|l|p{1.6cm}|p{1.5cm}|p{1.5cm}|p{1.5cm}|p{1.5cm}|}
  \hline
  \textbf{Problem} & \multicolumn{5}{c|}{\(p\)}\\
  \cline{2-6}
  &\(1\)(time, s) & \(2\) & \(4\) & \(8\) & \(16\)\\
  \hline
  Markin-Strongin & 104.47 & 1.97 & 3.65 & 6.79 & 9.90 \\
  \hline
  Fonseca and Fleming 2d & 118.95 & 1.85 & 2.81 & 5.79 & 6.40 \\
  \hline
  Fonseca and Fleming 3d & 554.45 & 1.51 & 4.14 & 8.05 & 10.69 \\
  \hline
  Viennet problem & 1488.6 & 2.22 & 3.64 & 6.98 & 13.49\\
  \hline
  Poloni's problem & 336.74 & 1.82 & 3.64 & 6.98 & 10.60 \\
  \hline
  \end{tabular}
  \end{table}
\end{frame}


\begin{frame}{{}}
  \frametitle{ }
  \begin{center}
    \Large{Q\&A}

\vspace{1cm}
    Vladislav Sovrasov

    sovrasov.vlad@gmail.com

    https://github.com/sovrasov
  \end{center}
\end{frame}

\end{document}
